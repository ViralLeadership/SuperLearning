
% Default to the notebook output style

    


% Inherit from the specified cell style.




    
\documentclass[11pt]{article}

    
    
    \usepackage[T1]{fontenc}
    % Nicer default font (+ math font) than Computer Modern for most use cases
    \usepackage{mathpazo}

    % Basic figure setup, for now with no caption control since it's done
    % automatically by Pandoc (which extracts ![](path) syntax from Markdown).
    \usepackage{graphicx}
    % We will generate all images so they have a width \maxwidth. This means
    % that they will get their normal width if they fit onto the page, but
    % are scaled down if they would overflow the margins.
    \makeatletter
    \def\maxwidth{\ifdim\Gin@nat@width>\linewidth\linewidth
    \else\Gin@nat@width\fi}
    \makeatother
    \let\Oldincludegraphics\includegraphics
    % Set max figure width to be 80% of text width, for now hardcoded.
    \renewcommand{\includegraphics}[1]{\Oldincludegraphics[width=.8\maxwidth]{#1}}
    % Ensure that by default, figures have no caption (until we provide a
    % proper Figure object with a Caption API and a way to capture that
    % in the conversion process - todo).
    \usepackage{caption}
    \DeclareCaptionLabelFormat{nolabel}{}
    \captionsetup{labelformat=nolabel}

    \usepackage{adjustbox} % Used to constrain images to a maximum size 
    \usepackage{xcolor} % Allow colors to be defined
    \usepackage{enumerate} % Needed for markdown enumerations to work
    \usepackage{geometry} % Used to adjust the document margins
    \usepackage{amsmath} % Equations
    \usepackage{amssymb} % Equations
    \usepackage{textcomp} % defines textquotesingle
    % Hack from http://tex.stackexchange.com/a/47451/13684:
    \AtBeginDocument{%
        \def\PYZsq{\textquotesingle}% Upright quotes in Pygmentized code
    }
    \usepackage{upquote} % Upright quotes for verbatim code
    \usepackage{eurosym} % defines \euro
    \usepackage[mathletters]{ucs} % Extended unicode (utf-8) support
    \usepackage[utf8x]{inputenc} % Allow utf-8 characters in the tex document
    \usepackage{fancyvrb} % verbatim replacement that allows latex
    \usepackage{grffile} % extends the file name processing of package graphics 
                         % to support a larger range 
    % The hyperref package gives us a pdf with properly built
    % internal navigation ('pdf bookmarks' for the table of contents,
    % internal cross-reference links, web links for URLs, etc.)
    \usepackage{hyperref}
    \usepackage{longtable} % longtable support required by pandoc >1.10
    \usepackage{booktabs}  % table support for pandoc > 1.12.2
    \usepackage[inline]{enumitem} % IRkernel/repr support (it uses the enumerate* environment)
    \usepackage[normalem]{ulem} % ulem is needed to support strikethroughs (\sout)
                                % normalem makes italics be italics, not underlines
    

    
    
    % Colors for the hyperref package
    \definecolor{urlcolor}{rgb}{0,.145,.698}
    \definecolor{linkcolor}{rgb}{.71,0.21,0.01}
    \definecolor{citecolor}{rgb}{.12,.54,.11}

    % ANSI colors
    \definecolor{ansi-black}{HTML}{3E424D}
    \definecolor{ansi-black-intense}{HTML}{282C36}
    \definecolor{ansi-red}{HTML}{E75C58}
    \definecolor{ansi-red-intense}{HTML}{B22B31}
    \definecolor{ansi-green}{HTML}{00A250}
    \definecolor{ansi-green-intense}{HTML}{007427}
    \definecolor{ansi-yellow}{HTML}{DDB62B}
    \definecolor{ansi-yellow-intense}{HTML}{B27D12}
    \definecolor{ansi-blue}{HTML}{208FFB}
    \definecolor{ansi-blue-intense}{HTML}{0065CA}
    \definecolor{ansi-magenta}{HTML}{D160C4}
    \definecolor{ansi-magenta-intense}{HTML}{A03196}
    \definecolor{ansi-cyan}{HTML}{60C6C8}
    \definecolor{ansi-cyan-intense}{HTML}{258F8F}
    \definecolor{ansi-white}{HTML}{C5C1B4}
    \definecolor{ansi-white-intense}{HTML}{A1A6B2}

    % commands and environments needed by pandoc snippets
    % extracted from the output of `pandoc -s`
    \providecommand{\tightlist}{%
      \setlength{\itemsep}{0pt}\setlength{\parskip}{0pt}}
    \DefineVerbatimEnvironment{Highlighting}{Verbatim}{commandchars=\\\{\}}
    % Add ',fontsize=\small' for more characters per line
    \newenvironment{Shaded}{}{}
    \newcommand{\KeywordTok}[1]{\textcolor[rgb]{0.00,0.44,0.13}{\textbf{{#1}}}}
    \newcommand{\DataTypeTok}[1]{\textcolor[rgb]{0.56,0.13,0.00}{{#1}}}
    \newcommand{\DecValTok}[1]{\textcolor[rgb]{0.25,0.63,0.44}{{#1}}}
    \newcommand{\BaseNTok}[1]{\textcolor[rgb]{0.25,0.63,0.44}{{#1}}}
    \newcommand{\FloatTok}[1]{\textcolor[rgb]{0.25,0.63,0.44}{{#1}}}
    \newcommand{\CharTok}[1]{\textcolor[rgb]{0.25,0.44,0.63}{{#1}}}
    \newcommand{\StringTok}[1]{\textcolor[rgb]{0.25,0.44,0.63}{{#1}}}
    \newcommand{\CommentTok}[1]{\textcolor[rgb]{0.38,0.63,0.69}{\textit{{#1}}}}
    \newcommand{\OtherTok}[1]{\textcolor[rgb]{0.00,0.44,0.13}{{#1}}}
    \newcommand{\AlertTok}[1]{\textcolor[rgb]{1.00,0.00,0.00}{\textbf{{#1}}}}
    \newcommand{\FunctionTok}[1]{\textcolor[rgb]{0.02,0.16,0.49}{{#1}}}
    \newcommand{\RegionMarkerTok}[1]{{#1}}
    \newcommand{\ErrorTok}[1]{\textcolor[rgb]{1.00,0.00,0.00}{\textbf{{#1}}}}
    \newcommand{\NormalTok}[1]{{#1}}
    
    % Additional commands for more recent versions of Pandoc
    \newcommand{\ConstantTok}[1]{\textcolor[rgb]{0.53,0.00,0.00}{{#1}}}
    \newcommand{\SpecialCharTok}[1]{\textcolor[rgb]{0.25,0.44,0.63}{{#1}}}
    \newcommand{\VerbatimStringTok}[1]{\textcolor[rgb]{0.25,0.44,0.63}{{#1}}}
    \newcommand{\SpecialStringTok}[1]{\textcolor[rgb]{0.73,0.40,0.53}{{#1}}}
    \newcommand{\ImportTok}[1]{{#1}}
    \newcommand{\DocumentationTok}[1]{\textcolor[rgb]{0.73,0.13,0.13}{\textit{{#1}}}}
    \newcommand{\AnnotationTok}[1]{\textcolor[rgb]{0.38,0.63,0.69}{\textbf{\textit{{#1}}}}}
    \newcommand{\CommentVarTok}[1]{\textcolor[rgb]{0.38,0.63,0.69}{\textbf{\textit{{#1}}}}}
    \newcommand{\VariableTok}[1]{\textcolor[rgb]{0.10,0.09,0.49}{{#1}}}
    \newcommand{\ControlFlowTok}[1]{\textcolor[rgb]{0.00,0.44,0.13}{\textbf{{#1}}}}
    \newcommand{\OperatorTok}[1]{\textcolor[rgb]{0.40,0.40,0.40}{{#1}}}
    \newcommand{\BuiltInTok}[1]{{#1}}
    \newcommand{\ExtensionTok}[1]{{#1}}
    \newcommand{\PreprocessorTok}[1]{\textcolor[rgb]{0.74,0.48,0.00}{{#1}}}
    \newcommand{\AttributeTok}[1]{\textcolor[rgb]{0.49,0.56,0.16}{{#1}}}
    \newcommand{\InformationTok}[1]{\textcolor[rgb]{0.38,0.63,0.69}{\textbf{\textit{{#1}}}}}
    \newcommand{\WarningTok}[1]{\textcolor[rgb]{0.38,0.63,0.69}{\textbf{\textit{{#1}}}}}
    
    
    % Define a nice break command that doesn't care if a line doesn't already
    % exist.
    \def\br{\hspace*{\fill} \\* }
    % Math Jax compatability definitions
    \def\gt{>}
    \def\lt{<}
    % Document parameters
    \title{JJenkins\_HW \#2}
    
    
    

    % Pygments definitions
    
\makeatletter
\def\PY@reset{\let\PY@it=\relax \let\PY@bf=\relax%
    \let\PY@ul=\relax \let\PY@tc=\relax%
    \let\PY@bc=\relax \let\PY@ff=\relax}
\def\PY@tok#1{\csname PY@tok@#1\endcsname}
\def\PY@toks#1+{\ifx\relax#1\empty\else%
    \PY@tok{#1}\expandafter\PY@toks\fi}
\def\PY@do#1{\PY@bc{\PY@tc{\PY@ul{%
    \PY@it{\PY@bf{\PY@ff{#1}}}}}}}
\def\PY#1#2{\PY@reset\PY@toks#1+\relax+\PY@do{#2}}

\expandafter\def\csname PY@tok@w\endcsname{\def\PY@tc##1{\textcolor[rgb]{0.73,0.73,0.73}{##1}}}
\expandafter\def\csname PY@tok@c\endcsname{\let\PY@it=\textit\def\PY@tc##1{\textcolor[rgb]{0.25,0.50,0.50}{##1}}}
\expandafter\def\csname PY@tok@cp\endcsname{\def\PY@tc##1{\textcolor[rgb]{0.74,0.48,0.00}{##1}}}
\expandafter\def\csname PY@tok@k\endcsname{\let\PY@bf=\textbf\def\PY@tc##1{\textcolor[rgb]{0.00,0.50,0.00}{##1}}}
\expandafter\def\csname PY@tok@kp\endcsname{\def\PY@tc##1{\textcolor[rgb]{0.00,0.50,0.00}{##1}}}
\expandafter\def\csname PY@tok@kt\endcsname{\def\PY@tc##1{\textcolor[rgb]{0.69,0.00,0.25}{##1}}}
\expandafter\def\csname PY@tok@o\endcsname{\def\PY@tc##1{\textcolor[rgb]{0.40,0.40,0.40}{##1}}}
\expandafter\def\csname PY@tok@ow\endcsname{\let\PY@bf=\textbf\def\PY@tc##1{\textcolor[rgb]{0.67,0.13,1.00}{##1}}}
\expandafter\def\csname PY@tok@nb\endcsname{\def\PY@tc##1{\textcolor[rgb]{0.00,0.50,0.00}{##1}}}
\expandafter\def\csname PY@tok@nf\endcsname{\def\PY@tc##1{\textcolor[rgb]{0.00,0.00,1.00}{##1}}}
\expandafter\def\csname PY@tok@nc\endcsname{\let\PY@bf=\textbf\def\PY@tc##1{\textcolor[rgb]{0.00,0.00,1.00}{##1}}}
\expandafter\def\csname PY@tok@nn\endcsname{\let\PY@bf=\textbf\def\PY@tc##1{\textcolor[rgb]{0.00,0.00,1.00}{##1}}}
\expandafter\def\csname PY@tok@ne\endcsname{\let\PY@bf=\textbf\def\PY@tc##1{\textcolor[rgb]{0.82,0.25,0.23}{##1}}}
\expandafter\def\csname PY@tok@nv\endcsname{\def\PY@tc##1{\textcolor[rgb]{0.10,0.09,0.49}{##1}}}
\expandafter\def\csname PY@tok@no\endcsname{\def\PY@tc##1{\textcolor[rgb]{0.53,0.00,0.00}{##1}}}
\expandafter\def\csname PY@tok@nl\endcsname{\def\PY@tc##1{\textcolor[rgb]{0.63,0.63,0.00}{##1}}}
\expandafter\def\csname PY@tok@ni\endcsname{\let\PY@bf=\textbf\def\PY@tc##1{\textcolor[rgb]{0.60,0.60,0.60}{##1}}}
\expandafter\def\csname PY@tok@na\endcsname{\def\PY@tc##1{\textcolor[rgb]{0.49,0.56,0.16}{##1}}}
\expandafter\def\csname PY@tok@nt\endcsname{\let\PY@bf=\textbf\def\PY@tc##1{\textcolor[rgb]{0.00,0.50,0.00}{##1}}}
\expandafter\def\csname PY@tok@nd\endcsname{\def\PY@tc##1{\textcolor[rgb]{0.67,0.13,1.00}{##1}}}
\expandafter\def\csname PY@tok@s\endcsname{\def\PY@tc##1{\textcolor[rgb]{0.73,0.13,0.13}{##1}}}
\expandafter\def\csname PY@tok@sd\endcsname{\let\PY@it=\textit\def\PY@tc##1{\textcolor[rgb]{0.73,0.13,0.13}{##1}}}
\expandafter\def\csname PY@tok@si\endcsname{\let\PY@bf=\textbf\def\PY@tc##1{\textcolor[rgb]{0.73,0.40,0.53}{##1}}}
\expandafter\def\csname PY@tok@se\endcsname{\let\PY@bf=\textbf\def\PY@tc##1{\textcolor[rgb]{0.73,0.40,0.13}{##1}}}
\expandafter\def\csname PY@tok@sr\endcsname{\def\PY@tc##1{\textcolor[rgb]{0.73,0.40,0.53}{##1}}}
\expandafter\def\csname PY@tok@ss\endcsname{\def\PY@tc##1{\textcolor[rgb]{0.10,0.09,0.49}{##1}}}
\expandafter\def\csname PY@tok@sx\endcsname{\def\PY@tc##1{\textcolor[rgb]{0.00,0.50,0.00}{##1}}}
\expandafter\def\csname PY@tok@m\endcsname{\def\PY@tc##1{\textcolor[rgb]{0.40,0.40,0.40}{##1}}}
\expandafter\def\csname PY@tok@gh\endcsname{\let\PY@bf=\textbf\def\PY@tc##1{\textcolor[rgb]{0.00,0.00,0.50}{##1}}}
\expandafter\def\csname PY@tok@gu\endcsname{\let\PY@bf=\textbf\def\PY@tc##1{\textcolor[rgb]{0.50,0.00,0.50}{##1}}}
\expandafter\def\csname PY@tok@gd\endcsname{\def\PY@tc##1{\textcolor[rgb]{0.63,0.00,0.00}{##1}}}
\expandafter\def\csname PY@tok@gi\endcsname{\def\PY@tc##1{\textcolor[rgb]{0.00,0.63,0.00}{##1}}}
\expandafter\def\csname PY@tok@gr\endcsname{\def\PY@tc##1{\textcolor[rgb]{1.00,0.00,0.00}{##1}}}
\expandafter\def\csname PY@tok@ge\endcsname{\let\PY@it=\textit}
\expandafter\def\csname PY@tok@gs\endcsname{\let\PY@bf=\textbf}
\expandafter\def\csname PY@tok@gp\endcsname{\let\PY@bf=\textbf\def\PY@tc##1{\textcolor[rgb]{0.00,0.00,0.50}{##1}}}
\expandafter\def\csname PY@tok@go\endcsname{\def\PY@tc##1{\textcolor[rgb]{0.53,0.53,0.53}{##1}}}
\expandafter\def\csname PY@tok@gt\endcsname{\def\PY@tc##1{\textcolor[rgb]{0.00,0.27,0.87}{##1}}}
\expandafter\def\csname PY@tok@err\endcsname{\def\PY@bc##1{\setlength{\fboxsep}{0pt}\fcolorbox[rgb]{1.00,0.00,0.00}{1,1,1}{\strut ##1}}}
\expandafter\def\csname PY@tok@kc\endcsname{\let\PY@bf=\textbf\def\PY@tc##1{\textcolor[rgb]{0.00,0.50,0.00}{##1}}}
\expandafter\def\csname PY@tok@kd\endcsname{\let\PY@bf=\textbf\def\PY@tc##1{\textcolor[rgb]{0.00,0.50,0.00}{##1}}}
\expandafter\def\csname PY@tok@kn\endcsname{\let\PY@bf=\textbf\def\PY@tc##1{\textcolor[rgb]{0.00,0.50,0.00}{##1}}}
\expandafter\def\csname PY@tok@kr\endcsname{\let\PY@bf=\textbf\def\PY@tc##1{\textcolor[rgb]{0.00,0.50,0.00}{##1}}}
\expandafter\def\csname PY@tok@bp\endcsname{\def\PY@tc##1{\textcolor[rgb]{0.00,0.50,0.00}{##1}}}
\expandafter\def\csname PY@tok@fm\endcsname{\def\PY@tc##1{\textcolor[rgb]{0.00,0.00,1.00}{##1}}}
\expandafter\def\csname PY@tok@vc\endcsname{\def\PY@tc##1{\textcolor[rgb]{0.10,0.09,0.49}{##1}}}
\expandafter\def\csname PY@tok@vg\endcsname{\def\PY@tc##1{\textcolor[rgb]{0.10,0.09,0.49}{##1}}}
\expandafter\def\csname PY@tok@vi\endcsname{\def\PY@tc##1{\textcolor[rgb]{0.10,0.09,0.49}{##1}}}
\expandafter\def\csname PY@tok@vm\endcsname{\def\PY@tc##1{\textcolor[rgb]{0.10,0.09,0.49}{##1}}}
\expandafter\def\csname PY@tok@sa\endcsname{\def\PY@tc##1{\textcolor[rgb]{0.73,0.13,0.13}{##1}}}
\expandafter\def\csname PY@tok@sb\endcsname{\def\PY@tc##1{\textcolor[rgb]{0.73,0.13,0.13}{##1}}}
\expandafter\def\csname PY@tok@sc\endcsname{\def\PY@tc##1{\textcolor[rgb]{0.73,0.13,0.13}{##1}}}
\expandafter\def\csname PY@tok@dl\endcsname{\def\PY@tc##1{\textcolor[rgb]{0.73,0.13,0.13}{##1}}}
\expandafter\def\csname PY@tok@s2\endcsname{\def\PY@tc##1{\textcolor[rgb]{0.73,0.13,0.13}{##1}}}
\expandafter\def\csname PY@tok@sh\endcsname{\def\PY@tc##1{\textcolor[rgb]{0.73,0.13,0.13}{##1}}}
\expandafter\def\csname PY@tok@s1\endcsname{\def\PY@tc##1{\textcolor[rgb]{0.73,0.13,0.13}{##1}}}
\expandafter\def\csname PY@tok@mb\endcsname{\def\PY@tc##1{\textcolor[rgb]{0.40,0.40,0.40}{##1}}}
\expandafter\def\csname PY@tok@mf\endcsname{\def\PY@tc##1{\textcolor[rgb]{0.40,0.40,0.40}{##1}}}
\expandafter\def\csname PY@tok@mh\endcsname{\def\PY@tc##1{\textcolor[rgb]{0.40,0.40,0.40}{##1}}}
\expandafter\def\csname PY@tok@mi\endcsname{\def\PY@tc##1{\textcolor[rgb]{0.40,0.40,0.40}{##1}}}
\expandafter\def\csname PY@tok@il\endcsname{\def\PY@tc##1{\textcolor[rgb]{0.40,0.40,0.40}{##1}}}
\expandafter\def\csname PY@tok@mo\endcsname{\def\PY@tc##1{\textcolor[rgb]{0.40,0.40,0.40}{##1}}}
\expandafter\def\csname PY@tok@ch\endcsname{\let\PY@it=\textit\def\PY@tc##1{\textcolor[rgb]{0.25,0.50,0.50}{##1}}}
\expandafter\def\csname PY@tok@cm\endcsname{\let\PY@it=\textit\def\PY@tc##1{\textcolor[rgb]{0.25,0.50,0.50}{##1}}}
\expandafter\def\csname PY@tok@cpf\endcsname{\let\PY@it=\textit\def\PY@tc##1{\textcolor[rgb]{0.25,0.50,0.50}{##1}}}
\expandafter\def\csname PY@tok@c1\endcsname{\let\PY@it=\textit\def\PY@tc##1{\textcolor[rgb]{0.25,0.50,0.50}{##1}}}
\expandafter\def\csname PY@tok@cs\endcsname{\let\PY@it=\textit\def\PY@tc##1{\textcolor[rgb]{0.25,0.50,0.50}{##1}}}

\def\PYZbs{\char`\\}
\def\PYZus{\char`\_}
\def\PYZob{\char`\{}
\def\PYZcb{\char`\}}
\def\PYZca{\char`\^}
\def\PYZam{\char`\&}
\def\PYZlt{\char`\<}
\def\PYZgt{\char`\>}
\def\PYZsh{\char`\#}
\def\PYZpc{\char`\%}
\def\PYZdl{\char`\$}
\def\PYZhy{\char`\-}
\def\PYZsq{\char`\'}
\def\PYZdq{\char`\"}
\def\PYZti{\char`\~}
% for compatibility with earlier versions
\def\PYZat{@}
\def\PYZlb{[}
\def\PYZrb{]}
\makeatother


    % Exact colors from NB
    \definecolor{incolor}{rgb}{0.0, 0.0, 0.5}
    \definecolor{outcolor}{rgb}{0.545, 0.0, 0.0}



    
    % Prevent overflowing lines due to hard-to-break entities
    \sloppy 
    % Setup hyperref package
    \hypersetup{
      breaklinks=true,  % so long urls are correctly broken across lines
      colorlinks=true,
      urlcolor=urlcolor,
      linkcolor=linkcolor,
      citecolor=citecolor,
      }
    % Slightly bigger margins than the latex defaults
    
    \geometry{verbose,tmargin=1in,bmargin=1in,lmargin=1in,rmargin=1in}
    
    

    \begin{document}
    
    
    \maketitle
    
    

    
    \hypertarget{jjenkins-assignment-2}{%
\section{JJenkins Assignment 2}\label{jjenkins-assignment-2}}

    \hypertarget{exercise-1-6-marks}{%
\subsection{Exercise 1 (6 Marks)}\label{exercise-1-6-marks}}

\begin{itemize}
\tightlist
\item
  Create a list with 10 string or name items and print the list
\item
  Remove the last 3 items from the list
\item
  Append 3 new items to the list
\item
  Insert 1 new item between items 4 and 5
\item
  Change item 2 to something different
\item
  Take a slice from the list and print it
\end{itemize}

Remember to print each result. You can do all items in one Python cell
if you want. Use remark (\#) statements to label each part of your
solution. If you need to separate the output for neatness, you can
insert an empty print() statement or use whitespace.

Use the code cell below for your solution.

    \begin{Verbatim}[commandchars=\\\{\}]
{\color{incolor}In [{\color{incolor}1}]:} \PY{n+nb}{print}\PY{p}{(}\PY{l+s+s2}{\PYZdq{}}\PY{l+s+s2}{•Create a list with 10 string or name items and print the list•}\PY{l+s+s2}{\PYZdq{}}\PY{p}{)}
        \PY{n}{my\PYZus{}furniture} \PY{o}{=} \PY{p}{[}\PY{l+s+s1}{\PYZsq{}}\PY{l+s+s1}{couch}\PY{l+s+s1}{\PYZsq{}}\PY{p}{,}\PY{l+s+s1}{\PYZsq{}}\PY{l+s+s1}{chair}\PY{l+s+s1}{\PYZsq{}}\PY{p}{,}\PY{l+s+s1}{\PYZsq{}}\PY{l+s+s1}{table}\PY{l+s+s1}{\PYZsq{}}\PY{p}{,}\PY{l+s+s1}{\PYZsq{}}\PY{l+s+s1}{desk}\PY{l+s+s1}{\PYZsq{}}\PY{p}{,} \PY{l+s+s1}{\PYZsq{}}\PY{l+s+s1}{stool}\PY{l+s+s1}{\PYZsq{}}\PY{p}{,}\PY{l+s+s1}{\PYZsq{}}\PY{l+s+s1}{bench}\PY{l+s+s1}{\PYZsq{}}\PY{p}{,} \PY{l+s+s1}{\PYZsq{}}\PY{l+s+s1}{bed}\PY{l+s+s1}{\PYZsq{}}\PY{p}{,} \PY{l+s+s1}{\PYZsq{}}\PY{l+s+s1}{hammock}\PY{l+s+s1}{\PYZsq{}}\PY{p}{,} \PY{l+s+s1}{\PYZsq{}}\PY{l+s+s1}{dresser}\PY{l+s+s1}{\PYZsq{}}\PY{p}{,}\PY{l+s+s1}{\PYZsq{}}\PY{l+s+s1}{ottoman}\PY{l+s+s1}{\PYZsq{}}\PY{p}{]}
        \PY{n+nb}{print}\PY{p}{(}\PY{l+s+s2}{\PYZdq{}}\PY{l+s+se}{\PYZbs{}n}\PY{l+s+s2}{My list of 10 strings is the following list of furniture}\PY{l+s+s2}{\PYZdq{}}\PY{p}{,} \PY{n}{my\PYZus{}furniture}\PY{p}{)}
        \PY{k}{for} \PY{n}{piece} \PY{o+ow}{in} \PY{n}{my\PYZus{}furniture}\PY{p}{:}
            \PY{n+nb}{print}\PY{p}{(}\PY{n}{piece}\PY{p}{)}
        
        \PY{n+nb}{print}\PY{p}{(}\PY{l+s+s2}{\PYZdq{}}\PY{l+s+se}{\PYZbs{}n}\PY{l+s+s2}{•Remove the last 3 items from the list•}\PY{l+s+s2}{\PYZdq{}}\PY{p}{)}
        \PY{n}{last\PYZus{}3} \PY{o}{=} \PY{n+nb}{range}\PY{p}{(}\PY{o}{\PYZhy{}}\PY{l+m+mi}{3}\PY{p}{,} \PY{l+m+mi}{0}\PY{p}{)}
        \PY{k}{for} \PY{n}{index} \PY{o+ow}{in} \PY{n}{last\PYZus{}3}\PY{p}{:}
            \PY{k}{del} \PY{n}{my\PYZus{}furniture}\PY{p}{[}\PY{n}{index}\PY{p}{]}
        \PY{n+nb}{print}\PY{p}{(}\PY{l+s+s2}{\PYZdq{}}\PY{l+s+se}{\PYZbs{}n}\PY{l+s+s2}{There are }\PY{l+s+s2}{\PYZdq{}} \PY{o}{+} \PY{n+nb}{str}\PY{p}{(}\PY{n+nb}{len}\PY{p}{(}\PY{n}{my\PYZus{}furniture}\PY{p}{)}\PY{p}{)} \PY{o}{+} \PY{l+s+s2}{\PYZdq{}}\PY{l+s+s2}{ items of furniture after I remove the last 3 items for the list.}\PY{l+s+s2}{\PYZdq{}}\PY{p}{)}
        \PY{n+nb}{print}\PY{p}{(}\PY{l+s+s2}{\PYZdq{}}\PY{l+s+s2}{My list of }\PY{l+s+s2}{\PYZdq{}} \PY{o}{+} \PY{n+nb}{str}\PY{p}{(}\PY{n+nb}{len}\PY{p}{(}\PY{n}{my\PYZus{}furniture}\PY{p}{)}\PY{p}{)} \PY{o}{+} \PY{l+s+s2}{\PYZdq{}}\PY{l+s+s2}{ items of furniture now includes the following:}\PY{l+s+s2}{\PYZdq{}} \PY{o}{+} \PY{l+s+s2}{\PYZdq{}}\PY{l+s+se}{\PYZbs{}n}\PY{l+s+s2}{\PYZdq{}}\PY{p}{,} \PY{n}{my\PYZus{}furniture}\PY{p}{)}
        \PY{k}{for} \PY{n}{piece} \PY{o+ow}{in} \PY{n}{my\PYZus{}furniture}\PY{p}{:}
            \PY{n+nb}{print}\PY{p}{(}\PY{n}{piece}\PY{p}{)}
        
        \PY{n+nb}{print}\PY{p}{(}\PY{l+s+s2}{\PYZdq{}}\PY{l+s+se}{\PYZbs{}n}\PY{l+s+s2}{•Append 3 new items to the list•}\PY{l+s+s2}{\PYZdq{}}\PY{p}{)}
        \PY{c+c1}{\PYZsh{} adding 3 new items to my list}
        \PY{n}{my\PYZus{}furniture}\PY{o}{.}\PY{n}{append}\PY{p}{(}\PY{l+s+s1}{\PYZsq{}}\PY{l+s+s1}{coat rack}\PY{l+s+s1}{\PYZsq{}}\PY{p}{)}
        \PY{n}{my\PYZus{}furniture}\PY{o}{.}\PY{n}{append}\PY{p}{(}\PY{l+s+s1}{\PYZsq{}}\PY{l+s+s1}{hutch}\PY{l+s+s1}{\PYZsq{}}\PY{p}{)}
        \PY{n}{my\PYZus{}furniture}\PY{o}{.}\PY{n}{append}\PY{p}{(}\PY{l+s+s1}{\PYZsq{}}\PY{l+s+s1}{book case}\PY{l+s+s1}{\PYZsq{}}\PY{p}{)}
        \PY{n+nb}{print}\PY{p}{(}\PY{l+s+s2}{\PYZdq{}}\PY{l+s+se}{\PYZbs{}n}\PY{l+s+s2}{My appended list of strings is the following }\PY{l+s+s2}{\PYZdq{}} \PY{o}{+} \PY{n+nb}{str}\PY{p}{(}\PY{n+nb}{len}\PY{p}{(}\PY{n}{my\PYZus{}furniture}\PY{p}{)}\PY{p}{)} 
              \PY{o}{+} \PY{l+s+s2}{\PYZdq{}}\PY{l+s+s2}{ item list of furniture:}\PY{l+s+s2}{\PYZdq{}} \PY{o}{+} \PY{l+s+s2}{\PYZdq{}}\PY{l+s+se}{\PYZbs{}n}\PY{l+s+s2}{\PYZdq{}}\PY{p}{,} \PY{n}{my\PYZus{}furniture}\PY{p}{)}
        \PY{k}{for} \PY{n}{piece} \PY{o+ow}{in} \PY{n}{my\PYZus{}furniture}\PY{p}{:}
            \PY{n+nb}{print}\PY{p}{(}\PY{n}{piece}\PY{p}{)}
        
        \PY{n+nb}{print}\PY{p}{(}\PY{l+s+s2}{\PYZdq{}}\PY{l+s+se}{\PYZbs{}n}\PY{l+s+s2}{•Insert 1 new item between items 4 and 5•}\PY{l+s+s2}{\PYZdq{}}\PY{p}{)}
        \PY{n}{my\PYZus{}furniture}\PY{o}{.}\PY{n}{insert}\PY{p}{(}\PY{l+m+mi}{4}\PY{p}{,} \PY{l+s+s1}{\PYZsq{}}\PY{l+s+s1}{pool table}\PY{l+s+s1}{\PYZsq{}}\PY{p}{)}
        \PY{n+nb}{print}\PY{p}{(}\PY{l+s+s2}{\PYZdq{}}\PY{l+s+s2}{Inserting a new item between items 4 and 5, my }\PY{l+s+s2}{\PYZdq{}} \PY{o}{+} \PY{n+nb}{str}\PY{p}{(}\PY{n+nb}{len}\PY{p}{(}\PY{n}{my\PYZus{}furniture}\PY{p}{)}\PY{p}{)} 
              \PY{o}{+} \PY{l+s+s2}{\PYZdq{}}\PY{l+s+s2}{ item list is now the following:}\PY{l+s+s2}{\PYZdq{}} \PY{l+s+s2}{\PYZdq{}}\PY{l+s+se}{\PYZbs{}n}\PY{l+s+s2}{\PYZdq{}}\PY{p}{,} \PY{n}{my\PYZus{}furniture}\PY{p}{)}
        \PY{k}{for} \PY{n}{piece} \PY{o+ow}{in} \PY{n}{my\PYZus{}furniture}\PY{p}{:}
            \PY{n+nb}{print}\PY{p}{(}\PY{n}{piece}\PY{p}{)}
        
        \PY{n+nb}{print}\PY{p}{(}\PY{l+s+s2}{\PYZdq{}}\PY{l+s+se}{\PYZbs{}n}\PY{l+s+s2}{•Change item 2 to something different•}\PY{l+s+s2}{\PYZdq{}}\PY{p}{)}
        \PY{n}{my\PYZus{}furniture}\PY{p}{[}\PY{l+m+mi}{1}\PY{p}{]} \PY{o}{=} \PY{l+s+s1}{\PYZsq{}}\PY{l+s+s1}{curio cabinet}\PY{l+s+s1}{\PYZsq{}}
        \PY{n+nb}{print}\PY{p}{(}\PY{l+s+s2}{\PYZdq{}}\PY{l+s+s2}{Changing item 2 to something different, my }\PY{l+s+s2}{\PYZdq{}} \PY{o}{+} \PY{n+nb}{str}\PY{p}{(}\PY{n+nb}{len}\PY{p}{(}\PY{n}{my\PYZus{}furniture}\PY{p}{)}\PY{p}{)} 
              \PY{o}{+} \PY{l+s+s2}{\PYZdq{}}\PY{l+s+s2}{ item list is now the following:}\PY{l+s+s2}{\PYZdq{}} \PY{l+s+s2}{\PYZdq{}}\PY{l+s+se}{\PYZbs{}n}\PY{l+s+s2}{\PYZdq{}}\PY{p}{,} \PY{n}{my\PYZus{}furniture}\PY{p}{)}
        \PY{k}{for} \PY{n}{piece} \PY{o+ow}{in} \PY{n}{my\PYZus{}furniture}\PY{p}{:}
            \PY{n+nb}{print}\PY{p}{(}\PY{n}{piece}\PY{p}{)}
        
        \PY{n+nb}{print}\PY{p}{(}\PY{l+s+s2}{\PYZdq{}}\PY{l+s+se}{\PYZbs{}n}\PY{l+s+s2}{•Take a slice from the list and print it•}\PY{l+s+s2}{\PYZdq{}}\PY{p}{)}
        \PY{n}{first\PYZus{}piece} \PY{o}{=} \PY{n}{my\PYZus{}furniture}\PY{p}{[}\PY{l+m+mi}{0}\PY{p}{]}
        \PY{n+nb}{print}\PY{p}{(}\PY{n}{my\PYZus{}furniture}\PY{p}{[}\PY{l+m+mi}{0}\PY{p}{]}\PY{p}{)}
        \PY{n+nb}{print}\PY{p}{(}\PY{l+s+s2}{\PYZdq{}}\PY{l+s+s2}{The first piece of furniture in my list is a }\PY{l+s+s2}{\PYZdq{}} \PY{o}{+} \PY{n}{first\PYZus{}piece} \PY{o}{+} \PY{p}{(}\PY{l+s+s2}{\PYZdq{}}\PY{l+s+s2}{.}\PY{l+s+se}{\PYZbs{}n}\PY{l+s+s2}{\PYZdq{}}\PY{p}{)}\PY{p}{)}
        \PY{n}{last\PYZus{}piece} \PY{o}{=} \PY{n}{my\PYZus{}furniture}\PY{p}{[}\PY{o}{\PYZhy{}}\PY{l+m+mi}{1}\PY{p}{]}
        \PY{n+nb}{print}\PY{p}{(}\PY{n}{last\PYZus{}piece}\PY{p}{)}
        \PY{n+nb}{print}\PY{p}{(}\PY{l+s+s2}{\PYZdq{}}\PY{l+s+s2}{The last piece of furniture in my list is a }\PY{l+s+s2}{\PYZdq{}} \PY{o}{+} \PY{n}{last\PYZus{}piece} \PY{o}{+} \PY{p}{(}\PY{l+s+s2}{\PYZdq{}}\PY{l+s+s2}{.}\PY{l+s+se}{\PYZbs{}n}\PY{l+s+s2}{\PYZdq{}}\PY{p}{)}\PY{p}{)}
        \PY{n+nb}{print}\PY{p}{(}\PY{l+s+s2}{\PYZdq{}}\PY{l+s+s2}{The middle pieces of furniture in my list include: }\PY{l+s+s2}{\PYZdq{}}\PY{p}{)}
        \PY{n+nb}{print}\PY{p}{(}\PY{n}{my\PYZus{}furniture}\PY{p}{[}\PY{l+m+mi}{1}\PY{p}{:}\PY{o}{\PYZhy{}}\PY{l+m+mi}{1}\PY{p}{]}\PY{p}{)}
        \PY{n+nb}{print}\PY{p}{(}\PY{p}{)}
\end{Verbatim}


    \begin{Verbatim}[commandchars=\\\{\}]
•Create a list with 10 string or name items and print the list•

My list of 10 strings is the following list of furniture ['couch', 'chair', 'table', 'desk', 'stool', 'bench', 'bed', 'hammock', 'dresser', 'ottoman']
couch
chair
table
desk
stool
bench
bed
hammock
dresser
ottoman

•Remove the last 3 items from the list•

There are 7 items of furniture after I remove the last 3 items for the list.
My list of 7 items of furniture now includes the following:
 ['couch', 'chair', 'table', 'desk', 'stool', 'bench', 'bed']
couch
chair
table
desk
stool
bench
bed

•Append 3 new items to the list•

My appended list of strings is the following 10 item list of furniture:
 ['couch', 'chair', 'table', 'desk', 'stool', 'bench', 'bed', 'coat rack', 'hutch', 'book case']
couch
chair
table
desk
stool
bench
bed
coat rack
hutch
book case

•Insert 1 new item between items 4 and 5•
Inserting a new item between items 4 and 5, my 11 item list is now the following:
 ['couch', 'chair', 'table', 'desk', 'pool table', 'stool', 'bench', 'bed', 'coat rack', 'hutch', 'book case']
couch
chair
table
desk
pool table
stool
bench
bed
coat rack
hutch
book case

•Change item 2 to something different•
Changing item 2 to something different, my 11 item list is now the following:
 ['couch', 'curio cabinet', 'table', 'desk', 'pool table', 'stool', 'bench', 'bed', 'coat rack', 'hutch', 'book case']
couch
curio cabinet
table
desk
pool table
stool
bench
bed
coat rack
hutch
book case

•Take a slice from the list and print it•
couch
The first piece of furniture in my list is a couch.

book case
The last piece of furniture in my list is a book case.

The middle pieces of furniture in my list include: 
['curio cabinet', 'table', 'desk', 'pool table', 'stool', 'bench', 'bed', 'coat rack', 'hutch']


    \end{Verbatim}

    \hypertarget{exercise-2-3-marks}{%
\subsection{Exercise 2 (3 Marks)}\label{exercise-2-3-marks}}

Write a program that calculates and prints the first 50 powers of 2
minus 1. Instead of using n**2, you will need the following function
(which we used earlier in the course):

2**n - 1

Hint: Use the squares program in Notebook 2 as a model.

Use the code cell below for your solution.

    \begin{Verbatim}[commandchars=\\\{\}]
{\color{incolor}In [{\color{incolor}2}]:} \PY{c+c1}{\PYZsh{} setting my n to zero and creating a variable for my exponent}
        \PY{n}{n} \PY{o}{=} \PY{l+m+mi}{0}
        \PY{n}{exponent} \PY{o}{=} \PY{n}{n}\PY{o}{\PYZhy{}}\PY{l+m+mi}{1}
        
        \PY{c+c1}{\PYZsh{} using a while loop to to loop through my counts of n from 0 to 50}
        \PY{k}{while} \PY{n}{n} \PY{o}{\PYZlt{}} \PY{l+m+mi}{51}\PY{p}{:}
            \PY{c+c1}{\PYZsh{} calculating 2 to the power of (n\PYZhy{}1) for each count of n from 0 to 50}
            \PY{n}{pwrs\PYZus{}of\PYZus{}2minus1} \PY{o}{=} \PY{l+m+mi}{2}\PY{o}{*}\PY{o}{*}\PY{n}{exponent}
            \PY{c+c1}{\PYZsh{} printing out the value of 2 to the power of (n\PYZhy{}1) for each count of n}
            \PY{n+nb}{print}\PY{p}{(}\PY{l+s+s2}{\PYZdq{}}\PY{l+s+s2}{2 to the power of (}\PY{l+s+s2}{\PYZdq{}} \PY{o}{+} \PY{n+nb}{str}\PY{p}{(}\PY{n}{n}\PY{p}{)} \PY{o}{+} \PY{l+s+s2}{\PYZdq{}}\PY{l+s+s2}{ minus 1), also written as 2\PYZca{}(}\PY{l+s+s2}{\PYZdq{}} \PY{o}{+} \PY{n+nb}{str}\PY{p}{(}\PY{n}{n}\PY{p}{)} \PY{o}{+} \PY{l+s+s2}{\PYZdq{}}\PY{l+s+s2}{\PYZhy{}1) is equal to }\PY{l+s+s2}{\PYZdq{}} 
                  \PY{o}{+} \PY{l+s+s2}{\PYZdq{}}\PY{l+s+s2}{2\PYZca{}(}\PY{l+s+s2}{\PYZdq{}} \PY{o}{+} \PY{n+nb}{str}\PY{p}{(}\PY{n}{exponent}\PY{p}{)} \PY{o}{+} \PY{l+s+s2}{\PYZdq{}}\PY{l+s+s2}{) and has a value of }\PY{l+s+s2}{\PYZdq{}} \PY{o}{+} \PY{n+nb}{str}\PY{p}{(}\PY{n}{pwrs\PYZus{}of\PYZus{}2minus1}\PY{p}{)} \PY{o}{+}\PY{l+s+s2}{\PYZdq{}}\PY{l+s+s2}{.}\PY{l+s+s2}{\PYZdq{}}\PY{p}{)}
            \PY{c+c1}{\PYZsh{} incrementing my count for this loop iteration to be passed to the next iteration}
            \PY{n}{n} \PY{o}{=} \PY{n}{n} \PY{o}{+} \PY{l+m+mi}{1}
            \PY{c+c1}{\PYZsh{} incrementing my exponent to be passed to the next iteration}
            \PY{n}{exponent} \PY{o}{=} \PY{n}{n} \PY{o}{\PYZhy{}}\PY{l+m+mi}{1}
\end{Verbatim}


    \begin{Verbatim}[commandchars=\\\{\}]
2 to the power of (0 minus 1), also written as 2\^{}(0-1) is equal to 2\^{}(-1) and has a value of 0.5.
2 to the power of (1 minus 1), also written as 2\^{}(1-1) is equal to 2\^{}(0) and has a value of 1.
2 to the power of (2 minus 1), also written as 2\^{}(2-1) is equal to 2\^{}(1) and has a value of 2.
2 to the power of (3 minus 1), also written as 2\^{}(3-1) is equal to 2\^{}(2) and has a value of 4.
2 to the power of (4 minus 1), also written as 2\^{}(4-1) is equal to 2\^{}(3) and has a value of 8.
2 to the power of (5 minus 1), also written as 2\^{}(5-1) is equal to 2\^{}(4) and has a value of 16.
2 to the power of (6 minus 1), also written as 2\^{}(6-1) is equal to 2\^{}(5) and has a value of 32.
2 to the power of (7 minus 1), also written as 2\^{}(7-1) is equal to 2\^{}(6) and has a value of 64.
2 to the power of (8 minus 1), also written as 2\^{}(8-1) is equal to 2\^{}(7) and has a value of 128.
2 to the power of (9 minus 1), also written as 2\^{}(9-1) is equal to 2\^{}(8) and has a value of 256.
2 to the power of (10 minus 1), also written as 2\^{}(10-1) is equal to 2\^{}(9) and has a value of 512.
2 to the power of (11 minus 1), also written as 2\^{}(11-1) is equal to 2\^{}(10) and has a value of 1024.
2 to the power of (12 minus 1), also written as 2\^{}(12-1) is equal to 2\^{}(11) and has a value of 2048.
2 to the power of (13 minus 1), also written as 2\^{}(13-1) is equal to 2\^{}(12) and has a value of 4096.
2 to the power of (14 minus 1), also written as 2\^{}(14-1) is equal to 2\^{}(13) and has a value of 8192.
2 to the power of (15 minus 1), also written as 2\^{}(15-1) is equal to 2\^{}(14) and has a value of 16384.
2 to the power of (16 minus 1), also written as 2\^{}(16-1) is equal to 2\^{}(15) and has a value of 32768.
2 to the power of (17 minus 1), also written as 2\^{}(17-1) is equal to 2\^{}(16) and has a value of 65536.
2 to the power of (18 minus 1), also written as 2\^{}(18-1) is equal to 2\^{}(17) and has a value of 131072.
2 to the power of (19 minus 1), also written as 2\^{}(19-1) is equal to 2\^{}(18) and has a value of 262144.
2 to the power of (20 minus 1), also written as 2\^{}(20-1) is equal to 2\^{}(19) and has a value of 524288.
2 to the power of (21 minus 1), also written as 2\^{}(21-1) is equal to 2\^{}(20) and has a value of 1048576.
2 to the power of (22 minus 1), also written as 2\^{}(22-1) is equal to 2\^{}(21) and has a value of 2097152.
2 to the power of (23 minus 1), also written as 2\^{}(23-1) is equal to 2\^{}(22) and has a value of 4194304.
2 to the power of (24 minus 1), also written as 2\^{}(24-1) is equal to 2\^{}(23) and has a value of 8388608.
2 to the power of (25 minus 1), also written as 2\^{}(25-1) is equal to 2\^{}(24) and has a value of 16777216.
2 to the power of (26 minus 1), also written as 2\^{}(26-1) is equal to 2\^{}(25) and has a value of 33554432.
2 to the power of (27 minus 1), also written as 2\^{}(27-1) is equal to 2\^{}(26) and has a value of 67108864.
2 to the power of (28 minus 1), also written as 2\^{}(28-1) is equal to 2\^{}(27) and has a value of 134217728.
2 to the power of (29 minus 1), also written as 2\^{}(29-1) is equal to 2\^{}(28) and has a value of 268435456.
2 to the power of (30 minus 1), also written as 2\^{}(30-1) is equal to 2\^{}(29) and has a value of 536870912.
2 to the power of (31 minus 1), also written as 2\^{}(31-1) is equal to 2\^{}(30) and has a value of 1073741824.
2 to the power of (32 minus 1), also written as 2\^{}(32-1) is equal to 2\^{}(31) and has a value of 2147483648.
2 to the power of (33 minus 1), also written as 2\^{}(33-1) is equal to 2\^{}(32) and has a value of 4294967296.
2 to the power of (34 minus 1), also written as 2\^{}(34-1) is equal to 2\^{}(33) and has a value of 8589934592.
2 to the power of (35 minus 1), also written as 2\^{}(35-1) is equal to 2\^{}(34) and has a value of 17179869184.
2 to the power of (36 minus 1), also written as 2\^{}(36-1) is equal to 2\^{}(35) and has a value of 34359738368.
2 to the power of (37 minus 1), also written as 2\^{}(37-1) is equal to 2\^{}(36) and has a value of 68719476736.
2 to the power of (38 minus 1), also written as 2\^{}(38-1) is equal to 2\^{}(37) and has a value of 137438953472.
2 to the power of (39 minus 1), also written as 2\^{}(39-1) is equal to 2\^{}(38) and has a value of 274877906944.
2 to the power of (40 minus 1), also written as 2\^{}(40-1) is equal to 2\^{}(39) and has a value of 549755813888.
2 to the power of (41 minus 1), also written as 2\^{}(41-1) is equal to 2\^{}(40) and has a value of 1099511627776.
2 to the power of (42 minus 1), also written as 2\^{}(42-1) is equal to 2\^{}(41) and has a value of 2199023255552.
2 to the power of (43 minus 1), also written as 2\^{}(43-1) is equal to 2\^{}(42) and has a value of 4398046511104.
2 to the power of (44 minus 1), also written as 2\^{}(44-1) is equal to 2\^{}(43) and has a value of 8796093022208.
2 to the power of (45 minus 1), also written as 2\^{}(45-1) is equal to 2\^{}(44) and has a value of 17592186044416.
2 to the power of (46 minus 1), also written as 2\^{}(46-1) is equal to 2\^{}(45) and has a value of 35184372088832.
2 to the power of (47 minus 1), also written as 2\^{}(47-1) is equal to 2\^{}(46) and has a value of 70368744177664.
2 to the power of (48 minus 1), also written as 2\^{}(48-1) is equal to 2\^{}(47) and has a value of 140737488355328.
2 to the power of (49 minus 1), also written as 2\^{}(49-1) is equal to 2\^{}(48) and has a value of 281474976710656.
2 to the power of (50 minus 1), also written as 2\^{}(50-1) is equal to 2\^{}(49) and has a value of 562949953421312.

    \end{Verbatim}

    \hypertarget{exercise-3-6-marks}{%
\subsection{Exercise 3 (6 Marks)}\label{exercise-3-6-marks}}

Write a program that finds the sum of the first 1000 even integers. Not
the evens from 1 to 1000, but the first \textbf{1000} even integers.
Also calculate the average of the 1000 even integers. Print the length
of the list, the sum of the list, the average of the list and the last 5
even numbers in the list.

    \begin{Verbatim}[commandchars=\\\{\}]
{\color{incolor}In [{\color{incolor}3}]:} \PY{c+c1}{\PYZsh{} creating an empty list for my even integers and initializing my count of even integers}
        \PY{n}{even\PYZus{}integers} \PY{o}{=} \PY{p}{[}\PY{p}{]}
        \PY{n}{even\PYZus{}integers\PYZus{}count} \PY{o}{=} \PY{l+m+mi}{0}
        
        \PY{c+c1}{\PYZsh{} using a while loop that will run until my list of even integers contains 1000 items}
        \PY{k}{while} \PY{n+nb}{len}\PY{p}{(}\PY{n}{even\PYZus{}integers}\PY{p}{)} \PY{o}{\PYZlt{}}\PY{o}{=} \PY{l+m+mi}{999}\PY{p}{:}
            \PY{c+c1}{\PYZsh{} using an if statement to check whether the current integer is even or not}
            \PY{k}{if} \PY{n}{even\PYZus{}integers\PYZus{}count} \PY{o}{\PYZpc{}} \PY{l+m+mi}{2} \PY{o}{==} \PY{l+m+mi}{0}\PY{p}{:}
                \PY{c+c1}{\PYZsh{} appending even integers to my list}
                \PY{n}{even\PYZus{}integers}\PY{o}{.}\PY{n}{append}\PY{p}{(}\PY{n}{even\PYZus{}integers\PYZus{}count}\PY{p}{)}
            \PY{c+c1}{\PYZsh{} incrementing my while loop to the next integer}
            \PY{n}{even\PYZus{}integers\PYZus{}count} \PY{o}{=} \PY{n}{even\PYZus{}integers\PYZus{}count} \PY{o}{+} \PY{l+m+mi}{1}
        
        
        \PY{c+c1}{\PYZsh{} printing the length of the list of even integers}
        \PY{n+nb}{print}\PY{p}{(}\PY{l+s+s2}{\PYZdq{}}\PY{l+s+s2}{The length of the list of even integers is }\PY{l+s+si}{\PYZob{}0:,g\PYZcb{}}\PY{l+s+s2}{\PYZdq{}}\PY{o}{.}\PY{n}{format}\PY{p}{(}\PY{n+nb}{len}\PY{p}{(}\PY{n}{even\PYZus{}integers}\PY{p}{)}\PY{p}{)} \PY{o}{+} \PY{l+s+s2}{\PYZdq{}}\PY{l+s+s2}{.}\PY{l+s+se}{\PYZbs{}n}\PY{l+s+s2}{\PYZdq{}}\PY{p}{)}
        
        \PY{c+c1}{\PYZsh{} calculating the sum of my list of even integers and printing out the sum.}
        \PY{n}{sum\PYZus{}even\PYZus{}integers} \PY{o}{=} \PY{n+nb}{sum}\PY{p}{(}\PY{n}{even\PYZus{}integers}\PY{p}{)}
        \PY{n+nb}{print}\PY{p}{(}\PY{l+s+s2}{\PYZdq{}}\PY{l+s+s2}{The sum of the first }\PY{l+s+si}{\PYZob{}0:,g\PYZcb{}}\PY{l+s+s2}{\PYZdq{}}\PY{o}{.}\PY{n}{format}\PY{p}{(}\PY{n+nb}{len}\PY{p}{(}\PY{n}{even\PYZus{}integers}\PY{p}{)}\PY{p}{)} 
              \PY{o}{+} \PY{l+s+s2}{\PYZdq{}}\PY{l+s+s2}{ even integers is equal to }\PY{l+s+si}{\PYZob{}0:,g\PYZcb{}}\PY{l+s+s2}{\PYZdq{}}\PY{o}{.}\PY{n}{format}\PY{p}{(}\PY{n}{sum\PYZus{}even\PYZus{}integers}\PY{p}{)} \PY{o}{+} \PY{l+s+s2}{\PYZdq{}}\PY{l+s+s2}{.}\PY{l+s+se}{\PYZbs{}n}\PY{l+s+s2}{\PYZdq{}}\PY{p}{)}
        
        \PY{c+c1}{\PYZsh{} calculating the average of my list of even integers and printing out the average.}
        \PY{n}{avg\PYZus{}even\PYZus{}integers} \PY{o}{=} \PY{n}{sum\PYZus{}even\PYZus{}integers} \PY{o}{/} \PY{n+nb}{float}\PY{p}{(}\PY{n+nb}{len}\PY{p}{(}\PY{n}{even\PYZus{}integers}\PY{p}{)}\PY{p}{)}
        \PY{n+nb}{print}\PY{p}{(}\PY{l+s+s2}{\PYZdq{}}\PY{l+s+s2}{The average of the first }\PY{l+s+si}{\PYZob{}0:,g\PYZcb{}}\PY{l+s+s2}{\PYZdq{}}\PY{o}{.}\PY{n}{format}\PY{p}{(}\PY{n+nb}{len}\PY{p}{(}\PY{n}{even\PYZus{}integers}\PY{p}{)}\PY{p}{)} 
              \PY{o}{+} \PY{l+s+s2}{\PYZdq{}}\PY{l+s+s2}{ even integers is equal to }\PY{l+s+si}{\PYZob{}0:,g\PYZcb{}}\PY{l+s+s2}{\PYZdq{}}\PY{o}{.}\PY{n}{format}\PY{p}{(}\PY{n}{avg\PYZus{}even\PYZus{}integers}\PY{p}{)} \PY{o}{+} \PY{l+s+s2}{\PYZdq{}}\PY{l+s+s2}{.}\PY{l+s+se}{\PYZbs{}n}\PY{l+s+s2}{\PYZdq{}}\PY{p}{)}
        
        \PY{c+c1}{\PYZsh{} printing out the last 5 numbers in the list}
        \PY{n+nb}{print}\PY{p}{(}\PY{l+s+s2}{\PYZdq{}}\PY{l+s+s2}{The last five numbers in the list are: }\PY{l+s+s2}{\PYZdq{}}\PY{p}{,} \PY{n}{even\PYZus{}integers}\PY{p}{[}\PY{o}{\PYZhy{}}\PY{l+m+mi}{5}\PY{p}{:}\PY{p}{]}\PY{p}{)}
\end{Verbatim}


    \begin{Verbatim}[commandchars=\\\{\}]
The length of the list of even integers is 1,000.

The sum of the first 1,000 even integers is equal to 999,000.

The average of the first 1,000 even integers is equal to 999.

The last five numbers in the list are:  [1990, 1992, 1994, 1996, 1998]

    \end{Verbatim}


    % Add a bibliography block to the postdoc
    
    
    
    \end{document}
