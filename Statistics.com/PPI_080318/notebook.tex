
% Default to the notebook output style

    


% Inherit from the specified cell style.




    
\documentclass[11pt]{article}

    
    
    \usepackage[T1]{fontenc}
    % Nicer default font (+ math font) than Computer Modern for most use cases
    \usepackage{mathpazo}

    % Basic figure setup, for now with no caption control since it's done
    % automatically by Pandoc (which extracts ![](path) syntax from Markdown).
    \usepackage{graphicx}
    % We will generate all images so they have a width \maxwidth. This means
    % that they will get their normal width if they fit onto the page, but
    % are scaled down if they would overflow the margins.
    \makeatletter
    \def\maxwidth{\ifdim\Gin@nat@width>\linewidth\linewidth
    \else\Gin@nat@width\fi}
    \makeatother
    \let\Oldincludegraphics\includegraphics
    % Set max figure width to be 80% of text width, for now hardcoded.
    \renewcommand{\includegraphics}[1]{\Oldincludegraphics[width=.8\maxwidth]{#1}}
    % Ensure that by default, figures have no caption (until we provide a
    % proper Figure object with a Caption API and a way to capture that
    % in the conversion process - todo).
    \usepackage{caption}
    \DeclareCaptionLabelFormat{nolabel}{}
    \captionsetup{labelformat=nolabel}

    \usepackage{adjustbox} % Used to constrain images to a maximum size 
    \usepackage{xcolor} % Allow colors to be defined
    \usepackage{enumerate} % Needed for markdown enumerations to work
    \usepackage{geometry} % Used to adjust the document margins
    \usepackage{amsmath} % Equations
    \usepackage{amssymb} % Equations
    \usepackage{textcomp} % defines textquotesingle
    % Hack from http://tex.stackexchange.com/a/47451/13684:
    \AtBeginDocument{%
        \def\PYZsq{\textquotesingle}% Upright quotes in Pygmentized code
    }
    \usepackage{upquote} % Upright quotes for verbatim code
    \usepackage{eurosym} % defines \euro
    \usepackage[mathletters]{ucs} % Extended unicode (utf-8) support
    \usepackage[utf8x]{inputenc} % Allow utf-8 characters in the tex document
    \usepackage{fancyvrb} % verbatim replacement that allows latex
    \usepackage{grffile} % extends the file name processing of package graphics 
                         % to support a larger range 
    % The hyperref package gives us a pdf with properly built
    % internal navigation ('pdf bookmarks' for the table of contents,
    % internal cross-reference links, web links for URLs, etc.)
    \usepackage{hyperref}
    \usepackage{longtable} % longtable support required by pandoc >1.10
    \usepackage{booktabs}  % table support for pandoc > 1.12.2
    \usepackage[inline]{enumitem} % IRkernel/repr support (it uses the enumerate* environment)
    \usepackage[normalem]{ulem} % ulem is needed to support strikethroughs (\sout)
                                % normalem makes italics be italics, not underlines
    

    
    
    % Colors for the hyperref package
    \definecolor{urlcolor}{rgb}{0,.145,.698}
    \definecolor{linkcolor}{rgb}{.71,0.21,0.01}
    \definecolor{citecolor}{rgb}{.12,.54,.11}

    % ANSI colors
    \definecolor{ansi-black}{HTML}{3E424D}
    \definecolor{ansi-black-intense}{HTML}{282C36}
    \definecolor{ansi-red}{HTML}{E75C58}
    \definecolor{ansi-red-intense}{HTML}{B22B31}
    \definecolor{ansi-green}{HTML}{00A250}
    \definecolor{ansi-green-intense}{HTML}{007427}
    \definecolor{ansi-yellow}{HTML}{DDB62B}
    \definecolor{ansi-yellow-intense}{HTML}{B27D12}
    \definecolor{ansi-blue}{HTML}{208FFB}
    \definecolor{ansi-blue-intense}{HTML}{0065CA}
    \definecolor{ansi-magenta}{HTML}{D160C4}
    \definecolor{ansi-magenta-intense}{HTML}{A03196}
    \definecolor{ansi-cyan}{HTML}{60C6C8}
    \definecolor{ansi-cyan-intense}{HTML}{258F8F}
    \definecolor{ansi-white}{HTML}{C5C1B4}
    \definecolor{ansi-white-intense}{HTML}{A1A6B2}

    % commands and environments needed by pandoc snippets
    % extracted from the output of `pandoc -s`
    \providecommand{\tightlist}{%
      \setlength{\itemsep}{0pt}\setlength{\parskip}{0pt}}
    \DefineVerbatimEnvironment{Highlighting}{Verbatim}{commandchars=\\\{\}}
    % Add ',fontsize=\small' for more characters per line
    \newenvironment{Shaded}{}{}
    \newcommand{\KeywordTok}[1]{\textcolor[rgb]{0.00,0.44,0.13}{\textbf{{#1}}}}
    \newcommand{\DataTypeTok}[1]{\textcolor[rgb]{0.56,0.13,0.00}{{#1}}}
    \newcommand{\DecValTok}[1]{\textcolor[rgb]{0.25,0.63,0.44}{{#1}}}
    \newcommand{\BaseNTok}[1]{\textcolor[rgb]{0.25,0.63,0.44}{{#1}}}
    \newcommand{\FloatTok}[1]{\textcolor[rgb]{0.25,0.63,0.44}{{#1}}}
    \newcommand{\CharTok}[1]{\textcolor[rgb]{0.25,0.44,0.63}{{#1}}}
    \newcommand{\StringTok}[1]{\textcolor[rgb]{0.25,0.44,0.63}{{#1}}}
    \newcommand{\CommentTok}[1]{\textcolor[rgb]{0.38,0.63,0.69}{\textit{{#1}}}}
    \newcommand{\OtherTok}[1]{\textcolor[rgb]{0.00,0.44,0.13}{{#1}}}
    \newcommand{\AlertTok}[1]{\textcolor[rgb]{1.00,0.00,0.00}{\textbf{{#1}}}}
    \newcommand{\FunctionTok}[1]{\textcolor[rgb]{0.02,0.16,0.49}{{#1}}}
    \newcommand{\RegionMarkerTok}[1]{{#1}}
    \newcommand{\ErrorTok}[1]{\textcolor[rgb]{1.00,0.00,0.00}{\textbf{{#1}}}}
    \newcommand{\NormalTok}[1]{{#1}}
    
    % Additional commands for more recent versions of Pandoc
    \newcommand{\ConstantTok}[1]{\textcolor[rgb]{0.53,0.00,0.00}{{#1}}}
    \newcommand{\SpecialCharTok}[1]{\textcolor[rgb]{0.25,0.44,0.63}{{#1}}}
    \newcommand{\VerbatimStringTok}[1]{\textcolor[rgb]{0.25,0.44,0.63}{{#1}}}
    \newcommand{\SpecialStringTok}[1]{\textcolor[rgb]{0.73,0.40,0.53}{{#1}}}
    \newcommand{\ImportTok}[1]{{#1}}
    \newcommand{\DocumentationTok}[1]{\textcolor[rgb]{0.73,0.13,0.13}{\textit{{#1}}}}
    \newcommand{\AnnotationTok}[1]{\textcolor[rgb]{0.38,0.63,0.69}{\textbf{\textit{{#1}}}}}
    \newcommand{\CommentVarTok}[1]{\textcolor[rgb]{0.38,0.63,0.69}{\textbf{\textit{{#1}}}}}
    \newcommand{\VariableTok}[1]{\textcolor[rgb]{0.10,0.09,0.49}{{#1}}}
    \newcommand{\ControlFlowTok}[1]{\textcolor[rgb]{0.00,0.44,0.13}{\textbf{{#1}}}}
    \newcommand{\OperatorTok}[1]{\textcolor[rgb]{0.40,0.40,0.40}{{#1}}}
    \newcommand{\BuiltInTok}[1]{{#1}}
    \newcommand{\ExtensionTok}[1]{{#1}}
    \newcommand{\PreprocessorTok}[1]{\textcolor[rgb]{0.74,0.48,0.00}{{#1}}}
    \newcommand{\AttributeTok}[1]{\textcolor[rgb]{0.49,0.56,0.16}{{#1}}}
    \newcommand{\InformationTok}[1]{\textcolor[rgb]{0.38,0.63,0.69}{\textbf{\textit{{#1}}}}}
    \newcommand{\WarningTok}[1]{\textcolor[rgb]{0.38,0.63,0.69}{\textbf{\textit{{#1}}}}}
    
    
    % Define a nice break command that doesn't care if a line doesn't already
    % exist.
    \def\br{\hspace*{\fill} \\* }
    % Math Jax compatability definitions
    \def\gt{>}
    \def\lt{<}
    % Document parameters
    \title{Week 3 Notebook 2 - While loops data entry and error trapping}
    
    
    

    % Pygments definitions
    
\makeatletter
\def\PY@reset{\let\PY@it=\relax \let\PY@bf=\relax%
    \let\PY@ul=\relax \let\PY@tc=\relax%
    \let\PY@bc=\relax \let\PY@ff=\relax}
\def\PY@tok#1{\csname PY@tok@#1\endcsname}
\def\PY@toks#1+{\ifx\relax#1\empty\else%
    \PY@tok{#1}\expandafter\PY@toks\fi}
\def\PY@do#1{\PY@bc{\PY@tc{\PY@ul{%
    \PY@it{\PY@bf{\PY@ff{#1}}}}}}}
\def\PY#1#2{\PY@reset\PY@toks#1+\relax+\PY@do{#2}}

\expandafter\def\csname PY@tok@w\endcsname{\def\PY@tc##1{\textcolor[rgb]{0.73,0.73,0.73}{##1}}}
\expandafter\def\csname PY@tok@c\endcsname{\let\PY@it=\textit\def\PY@tc##1{\textcolor[rgb]{0.25,0.50,0.50}{##1}}}
\expandafter\def\csname PY@tok@cp\endcsname{\def\PY@tc##1{\textcolor[rgb]{0.74,0.48,0.00}{##1}}}
\expandafter\def\csname PY@tok@k\endcsname{\let\PY@bf=\textbf\def\PY@tc##1{\textcolor[rgb]{0.00,0.50,0.00}{##1}}}
\expandafter\def\csname PY@tok@kp\endcsname{\def\PY@tc##1{\textcolor[rgb]{0.00,0.50,0.00}{##1}}}
\expandafter\def\csname PY@tok@kt\endcsname{\def\PY@tc##1{\textcolor[rgb]{0.69,0.00,0.25}{##1}}}
\expandafter\def\csname PY@tok@o\endcsname{\def\PY@tc##1{\textcolor[rgb]{0.40,0.40,0.40}{##1}}}
\expandafter\def\csname PY@tok@ow\endcsname{\let\PY@bf=\textbf\def\PY@tc##1{\textcolor[rgb]{0.67,0.13,1.00}{##1}}}
\expandafter\def\csname PY@tok@nb\endcsname{\def\PY@tc##1{\textcolor[rgb]{0.00,0.50,0.00}{##1}}}
\expandafter\def\csname PY@tok@nf\endcsname{\def\PY@tc##1{\textcolor[rgb]{0.00,0.00,1.00}{##1}}}
\expandafter\def\csname PY@tok@nc\endcsname{\let\PY@bf=\textbf\def\PY@tc##1{\textcolor[rgb]{0.00,0.00,1.00}{##1}}}
\expandafter\def\csname PY@tok@nn\endcsname{\let\PY@bf=\textbf\def\PY@tc##1{\textcolor[rgb]{0.00,0.00,1.00}{##1}}}
\expandafter\def\csname PY@tok@ne\endcsname{\let\PY@bf=\textbf\def\PY@tc##1{\textcolor[rgb]{0.82,0.25,0.23}{##1}}}
\expandafter\def\csname PY@tok@nv\endcsname{\def\PY@tc##1{\textcolor[rgb]{0.10,0.09,0.49}{##1}}}
\expandafter\def\csname PY@tok@no\endcsname{\def\PY@tc##1{\textcolor[rgb]{0.53,0.00,0.00}{##1}}}
\expandafter\def\csname PY@tok@nl\endcsname{\def\PY@tc##1{\textcolor[rgb]{0.63,0.63,0.00}{##1}}}
\expandafter\def\csname PY@tok@ni\endcsname{\let\PY@bf=\textbf\def\PY@tc##1{\textcolor[rgb]{0.60,0.60,0.60}{##1}}}
\expandafter\def\csname PY@tok@na\endcsname{\def\PY@tc##1{\textcolor[rgb]{0.49,0.56,0.16}{##1}}}
\expandafter\def\csname PY@tok@nt\endcsname{\let\PY@bf=\textbf\def\PY@tc##1{\textcolor[rgb]{0.00,0.50,0.00}{##1}}}
\expandafter\def\csname PY@tok@nd\endcsname{\def\PY@tc##1{\textcolor[rgb]{0.67,0.13,1.00}{##1}}}
\expandafter\def\csname PY@tok@s\endcsname{\def\PY@tc##1{\textcolor[rgb]{0.73,0.13,0.13}{##1}}}
\expandafter\def\csname PY@tok@sd\endcsname{\let\PY@it=\textit\def\PY@tc##1{\textcolor[rgb]{0.73,0.13,0.13}{##1}}}
\expandafter\def\csname PY@tok@si\endcsname{\let\PY@bf=\textbf\def\PY@tc##1{\textcolor[rgb]{0.73,0.40,0.53}{##1}}}
\expandafter\def\csname PY@tok@se\endcsname{\let\PY@bf=\textbf\def\PY@tc##1{\textcolor[rgb]{0.73,0.40,0.13}{##1}}}
\expandafter\def\csname PY@tok@sr\endcsname{\def\PY@tc##1{\textcolor[rgb]{0.73,0.40,0.53}{##1}}}
\expandafter\def\csname PY@tok@ss\endcsname{\def\PY@tc##1{\textcolor[rgb]{0.10,0.09,0.49}{##1}}}
\expandafter\def\csname PY@tok@sx\endcsname{\def\PY@tc##1{\textcolor[rgb]{0.00,0.50,0.00}{##1}}}
\expandafter\def\csname PY@tok@m\endcsname{\def\PY@tc##1{\textcolor[rgb]{0.40,0.40,0.40}{##1}}}
\expandafter\def\csname PY@tok@gh\endcsname{\let\PY@bf=\textbf\def\PY@tc##1{\textcolor[rgb]{0.00,0.00,0.50}{##1}}}
\expandafter\def\csname PY@tok@gu\endcsname{\let\PY@bf=\textbf\def\PY@tc##1{\textcolor[rgb]{0.50,0.00,0.50}{##1}}}
\expandafter\def\csname PY@tok@gd\endcsname{\def\PY@tc##1{\textcolor[rgb]{0.63,0.00,0.00}{##1}}}
\expandafter\def\csname PY@tok@gi\endcsname{\def\PY@tc##1{\textcolor[rgb]{0.00,0.63,0.00}{##1}}}
\expandafter\def\csname PY@tok@gr\endcsname{\def\PY@tc##1{\textcolor[rgb]{1.00,0.00,0.00}{##1}}}
\expandafter\def\csname PY@tok@ge\endcsname{\let\PY@it=\textit}
\expandafter\def\csname PY@tok@gs\endcsname{\let\PY@bf=\textbf}
\expandafter\def\csname PY@tok@gp\endcsname{\let\PY@bf=\textbf\def\PY@tc##1{\textcolor[rgb]{0.00,0.00,0.50}{##1}}}
\expandafter\def\csname PY@tok@go\endcsname{\def\PY@tc##1{\textcolor[rgb]{0.53,0.53,0.53}{##1}}}
\expandafter\def\csname PY@tok@gt\endcsname{\def\PY@tc##1{\textcolor[rgb]{0.00,0.27,0.87}{##1}}}
\expandafter\def\csname PY@tok@err\endcsname{\def\PY@bc##1{\setlength{\fboxsep}{0pt}\fcolorbox[rgb]{1.00,0.00,0.00}{1,1,1}{\strut ##1}}}
\expandafter\def\csname PY@tok@kc\endcsname{\let\PY@bf=\textbf\def\PY@tc##1{\textcolor[rgb]{0.00,0.50,0.00}{##1}}}
\expandafter\def\csname PY@tok@kd\endcsname{\let\PY@bf=\textbf\def\PY@tc##1{\textcolor[rgb]{0.00,0.50,0.00}{##1}}}
\expandafter\def\csname PY@tok@kn\endcsname{\let\PY@bf=\textbf\def\PY@tc##1{\textcolor[rgb]{0.00,0.50,0.00}{##1}}}
\expandafter\def\csname PY@tok@kr\endcsname{\let\PY@bf=\textbf\def\PY@tc##1{\textcolor[rgb]{0.00,0.50,0.00}{##1}}}
\expandafter\def\csname PY@tok@bp\endcsname{\def\PY@tc##1{\textcolor[rgb]{0.00,0.50,0.00}{##1}}}
\expandafter\def\csname PY@tok@fm\endcsname{\def\PY@tc##1{\textcolor[rgb]{0.00,0.00,1.00}{##1}}}
\expandafter\def\csname PY@tok@vc\endcsname{\def\PY@tc##1{\textcolor[rgb]{0.10,0.09,0.49}{##1}}}
\expandafter\def\csname PY@tok@vg\endcsname{\def\PY@tc##1{\textcolor[rgb]{0.10,0.09,0.49}{##1}}}
\expandafter\def\csname PY@tok@vi\endcsname{\def\PY@tc##1{\textcolor[rgb]{0.10,0.09,0.49}{##1}}}
\expandafter\def\csname PY@tok@vm\endcsname{\def\PY@tc##1{\textcolor[rgb]{0.10,0.09,0.49}{##1}}}
\expandafter\def\csname PY@tok@sa\endcsname{\def\PY@tc##1{\textcolor[rgb]{0.73,0.13,0.13}{##1}}}
\expandafter\def\csname PY@tok@sb\endcsname{\def\PY@tc##1{\textcolor[rgb]{0.73,0.13,0.13}{##1}}}
\expandafter\def\csname PY@tok@sc\endcsname{\def\PY@tc##1{\textcolor[rgb]{0.73,0.13,0.13}{##1}}}
\expandafter\def\csname PY@tok@dl\endcsname{\def\PY@tc##1{\textcolor[rgb]{0.73,0.13,0.13}{##1}}}
\expandafter\def\csname PY@tok@s2\endcsname{\def\PY@tc##1{\textcolor[rgb]{0.73,0.13,0.13}{##1}}}
\expandafter\def\csname PY@tok@sh\endcsname{\def\PY@tc##1{\textcolor[rgb]{0.73,0.13,0.13}{##1}}}
\expandafter\def\csname PY@tok@s1\endcsname{\def\PY@tc##1{\textcolor[rgb]{0.73,0.13,0.13}{##1}}}
\expandafter\def\csname PY@tok@mb\endcsname{\def\PY@tc##1{\textcolor[rgb]{0.40,0.40,0.40}{##1}}}
\expandafter\def\csname PY@tok@mf\endcsname{\def\PY@tc##1{\textcolor[rgb]{0.40,0.40,0.40}{##1}}}
\expandafter\def\csname PY@tok@mh\endcsname{\def\PY@tc##1{\textcolor[rgb]{0.40,0.40,0.40}{##1}}}
\expandafter\def\csname PY@tok@mi\endcsname{\def\PY@tc##1{\textcolor[rgb]{0.40,0.40,0.40}{##1}}}
\expandafter\def\csname PY@tok@il\endcsname{\def\PY@tc##1{\textcolor[rgb]{0.40,0.40,0.40}{##1}}}
\expandafter\def\csname PY@tok@mo\endcsname{\def\PY@tc##1{\textcolor[rgb]{0.40,0.40,0.40}{##1}}}
\expandafter\def\csname PY@tok@ch\endcsname{\let\PY@it=\textit\def\PY@tc##1{\textcolor[rgb]{0.25,0.50,0.50}{##1}}}
\expandafter\def\csname PY@tok@cm\endcsname{\let\PY@it=\textit\def\PY@tc##1{\textcolor[rgb]{0.25,0.50,0.50}{##1}}}
\expandafter\def\csname PY@tok@cpf\endcsname{\let\PY@it=\textit\def\PY@tc##1{\textcolor[rgb]{0.25,0.50,0.50}{##1}}}
\expandafter\def\csname PY@tok@c1\endcsname{\let\PY@it=\textit\def\PY@tc##1{\textcolor[rgb]{0.25,0.50,0.50}{##1}}}
\expandafter\def\csname PY@tok@cs\endcsname{\let\PY@it=\textit\def\PY@tc##1{\textcolor[rgb]{0.25,0.50,0.50}{##1}}}

\def\PYZbs{\char`\\}
\def\PYZus{\char`\_}
\def\PYZob{\char`\{}
\def\PYZcb{\char`\}}
\def\PYZca{\char`\^}
\def\PYZam{\char`\&}
\def\PYZlt{\char`\<}
\def\PYZgt{\char`\>}
\def\PYZsh{\char`\#}
\def\PYZpc{\char`\%}
\def\PYZdl{\char`\$}
\def\PYZhy{\char`\-}
\def\PYZsq{\char`\'}
\def\PYZdq{\char`\"}
\def\PYZti{\char`\~}
% for compatibility with earlier versions
\def\PYZat{@}
\def\PYZlb{[}
\def\PYZrb{]}
\makeatother


    % Exact colors from NB
    \definecolor{incolor}{rgb}{0.0, 0.0, 0.5}
    \definecolor{outcolor}{rgb}{0.545, 0.0, 0.0}



    
    % Prevent overflowing lines due to hard-to-break entities
    \sloppy 
    % Setup hyperref package
    \hypersetup{
      breaklinks=true,  % so long urls are correctly broken across lines
      colorlinks=true,
      urlcolor=urlcolor,
      linkcolor=linkcolor,
      citecolor=citecolor,
      }
    % Slightly bigger margins than the latex defaults
    
    \geometry{verbose,tmargin=1in,bmargin=1in,lmargin=1in,rmargin=1in}
    
    

    \begin{document}
    
    
    \maketitle
    
    

    
    \hypertarget{while-loops-data-input-and-basic-error-trapping}{%
\section{While Loops, Data Input and Basic Error
Trapping}\label{while-loops-data-input-and-basic-error-trapping}}

As you might guess, while loops are another form of looping. While (no
pun intended) for loops have a definite beginning and end based on the
parameters in the range() statement, while loops are based on
conditionals much like if statements. As long as a specific condition is
True, the while loop will continue looping. When the condition becomes
False, the while loop ends. Here is a template for while loops:

\begin{Shaded}
\begin{Highlighting}[]
\CommentTok{# specify and set the initial condition to True}
\NormalTok{running }\OperatorTok{=} \VariableTok{True}

\CommentTok{# here is the while loop}
\ControlFlowTok{while}\NormalTok{ running:}
    \CommentTok{# do some stuff here}
    \CommentTok{# and do some stuff here if needed}
    \CommentTok{# at some point you would need}
    \CommentTok{# to set running = False}
    \CommentTok{# otherwise the loop will never end}
    
\CommentTok{# once the loop ends, the program continues here}
\end{Highlighting}
\end{Shaded}

\begin{itemize}
\tightlist
\item
  The while loop needs to start with a True condition
\item
  The while statement itself contains the condition test
\item
  As long as the condition remains True, the loop continues
\item
  As soon as the condition changes to False, the loop ends
\item
  Once the loop ends, the program continues with the lines after the
  loop
\end{itemize}

One common error is to create a loop where the condition NEVER becomes
False. In that event, the loop continues. We call this an infinite
loop\ldots{} or as my math professor friend calls it (and my favorite
expression), an eternal loop. You can usually break out of eternal loops
by using Ctrl-C or clicking on a Stop button in Jupyter (the black
square in the icon menu) or Cancel Build in Sublime Text or something
similar.

    \begin{Verbatim}[commandchars=\\\{\}]
{\color{incolor}In [{\color{incolor} }]:} \PY{c+c1}{\PYZsh{} here is a countdown}
        \PY{n}{count} \PY{o}{=} \PY{l+m+mi}{10}
        
        \PY{c+c1}{\PYZsh{} while loop with condition count \PYZgt{} 0}
        \PY{k}{while} \PY{n}{count} \PY{o}{\PYZgt{}} \PY{l+m+mi}{0}\PY{p}{:}
            \PY{n+nb}{print}\PY{p}{(}\PY{n}{count}\PY{p}{)}
            
            \PY{c+c1}{\PYZsh{} this is the decrement function}
            \PY{n}{count} \PY{o}{\PYZhy{}}\PY{o}{=} \PY{l+m+mi}{1}
            
        \PY{c+c1}{\PYZsh{} this line prints after the loop is over, i.e. count = 0}
        \PY{n+nb}{print}\PY{p}{(}\PY{l+s+s2}{\PYZdq{}}\PY{l+s+se}{\PYZbs{}n}\PY{l+s+s2}{Blast Off!}\PY{l+s+s2}{\PYZdq{}}\PY{p}{)}
            
\end{Verbatim}


    \hypertarget{counting-forwards-and-backwards}{%
\subsubsection{Counting forwards and
backwards}\label{counting-forwards-and-backwards}}

I probably should not have used the variable count, but it seemed
descriptive. Any variable will work!

The count -= 1 statement subtracts 1 from the variable count each time
the statement is executed. Counters are very useful in programming. You
can count forward by 1 by using count += 1. You can count by any value,
even non-integers if you wish. This method is not limited to addition
and subtraction. You can also use * and / if you wish by using count *=
(for example) or count /= 2.

You can also count by using:

\begin{Shaded}
\begin{Highlighting}[]
\NormalTok{count }\OperatorTok{=}\NormalTok{ count }\OperatorTok{+} \DecValTok{1}
\end{Highlighting}
\end{Shaded}

This seems like an odd statement! How can count be equal to itself plus
1? However, the = sign is the assignment symbol (the == is the logical
equals). You read this statement from right to left. The value of the
right side is assigned to the left side variable. Which count version is
best? I think the += symbol is a bit more efficient, but it's your
choice.

    \hypertarget{try-it-yourself}{%
\subsubsection{Try It Yourself}\label{try-it-yourself}}

Let's have some fun. Copy the blastoff program into the code cell below.
Then\ldots{}

\begin{itemize}
\tightlist
\item
  Change the count -= 1 to count -= 0.1. Does the program still work?
  What do you notice?
\item
  Change the count -= 1 to count *= 2. Be ready to click the square stop
  button! Look at the output (scroll down if needed)
\item
  Change the count -= 1 to count /= 2. Let the program run. What is the
  smallest value you see? After several decimal values, the remaining
  numbers are in scientific notation. For example `1.23889e-26' is the
  same as \[1.23889 \times 10^{-26}\] and 1.5678e+28 is the same as
  \[1.5678 \times 10^{28}\]
\end{itemize}

    \hypertarget{input-data-from-the-keyboard-and-trapping-input-errors}{%
\subsection{Input data from the keyboard and trapping input
errors}\label{input-data-from-the-keyboard-and-trapping-input-errors}}

There are times when you need some interaction with the user or
customer. This is easy to accomplish by using the the input() statement.
Here is an example:

    \begin{Verbatim}[commandchars=\\\{\}]
{\color{incolor}In [{\color{incolor} }]:} \PY{c+c1}{\PYZsh{} have the user type in an integer between 1 and 10 inclusive}
        \PY{n}{value} \PY{o}{=} \PY{n+nb}{input}\PY{p}{(}\PY{l+s+s2}{\PYZdq{}}\PY{l+s+s2}{Please type in an integer between 1 and 10 inclusive }\PY{l+s+s2}{\PYZdq{}}\PY{p}{)}
        
        \PY{c+c1}{\PYZsh{} print the value}
        \PY{n+nb}{print}\PY{p}{(}\PY{l+s+s2}{\PYZdq{}}\PY{l+s+s2}{You typed in }\PY{l+s+s2}{\PYZdq{}} \PY{o}{+} \PY{n+nb}{str}\PY{p}{(}\PY{n}{value}\PY{p}{)}\PY{p}{)}
\end{Verbatim}


    That's nice, but what prevents a user from typing in a value outside the
specified range\ldots{} or from entering a floating point value? It's
important to provide data entry error checking in all programs. Error
checking helps prevent conditions such as dividing by zero (always a bad
thing!) and other garbage input that can lead to garbage output. Let's
give this concept a try\ldots{} maybe a while loop will help?

Error trapping is an intermediate to advanced topic, but let's take a
look. There are a couple of types of error trapping going on in the next
program. First, the new keywords try: and except:

\begin{Shaded}
\begin{Highlighting}[]

    \ControlFlowTok{try}\NormalTok{:}
        \CommentTok{# try out the python statement to see if it generates}
        \CommentTok{# an error (or throws and exception... same thing)}
    \ControlFlowTok{except}\NormalTok{:}
        \CommentTok{# if the statement in the try: block causes an error or throws an exception}
        \CommentTok{# instead of halting the program and showing the error, go to this block}
        \CommentTok{# of code instead}
\end{Highlighting}
\end{Shaded}

The try: and except: blocks are very useful for catching errors,
particularly errors that may occur at random or in unexpected instances
(such as accidently dividing by zero).

The second error trap is the

\begin{Shaded}
\begin{Highlighting}[]
\ControlFlowTok{if}\NormalTok{ value }\OperatorTok{>=} \DecValTok{1} \KeywordTok{and}\NormalTok{ value }\OperatorTok{<=} \DecValTok{10}\NormalTok{:}
\end{Highlighting}
\end{Shaded}

statement in the try: block. If the first line in the try: block is OK,
that means the value is an integer. We then can test to see if the
integer is in the specified range.

Try running the program several times and test the error trapping. The
error trapping keeps the while loop working until the user types in the
correct input.

\hypertarget{note-if-the-program-seems-to-hang-or-does-not-accept-input-check-to-see-if-there-is-an-asterisk-in-the-bracket-next-to-the-in-preceding-the-code-block-something-like-in.-if-so-you-need-to-click-the-black-stop-button-or-select-the-kernel-menu-and-interrupt-or-any-of-the-restart-entries-most-likely-restart-clear-output-whatever-it-takes-to-clear-the-asterisk-and-return-control-back-to-you-so-you-can-run-the-program-again.}{%
\paragraph{Note: If the program seems to hang or does not accept input,
check to see if there is an asterisk in the bracket next to the In
preceding the code block\ldots{} something like In{[}*{]}. If so, you
need to click the black stop button or select the Kernel menu and
Interrupt or any of the Restart entries (most likely Restart \& Clear
Output)\ldots{} whatever it takes to clear the asterisk and return
control back to you so you can run the program
again.}\label{note-if-the-program-seems-to-hang-or-does-not-accept-input-check-to-see-if-there-is-an-asterisk-in-the-bracket-next-to-the-in-preceding-the-code-block-something-like-in.-if-so-you-need-to-click-the-black-stop-button-or-select-the-kernel-menu-and-interrupt-or-any-of-the-restart-entries-most-likely-restart-clear-output-whatever-it-takes-to-clear-the-asterisk-and-return-control-back-to-you-so-you-can-run-the-program-again.}}

    \begin{Verbatim}[commandchars=\\\{\}]
{\color{incolor}In [{\color{incolor} }]:} \PY{c+c1}{\PYZsh{} here is the condition}
        \PY{n}{key\PYZus{}in} \PY{o}{=} \PY{k+kc}{True}
        
        \PY{k}{while} \PY{n}{key\PYZus{}in}\PY{p}{:}
            
            \PY{c+c1}{\PYZsh{} trap input errors using try and except}
            \PY{k}{try}\PY{p}{:}
                \PY{c+c1}{\PYZsh{} this statement checks to make certain the entered data is an integer}
                \PY{n}{value} \PY{o}{=} \PY{n+nb}{int}\PY{p}{(}\PY{n+nb}{input}\PY{p}{(}\PY{l+s+s2}{\PYZdq{}}\PY{l+s+s2}{Please type in an integer between 1 and 10 inclusive }\PY{l+s+s2}{\PYZdq{}}\PY{p}{)}\PY{p}{)}
                
                \PY{c+c1}{\PYZsh{} at this point, we know we have an integer... but is it in the range?}
                \PY{k}{if} \PY{n}{value} \PY{o}{\PYZgt{}}\PY{o}{=} \PY{l+m+mi}{1} \PY{o+ow}{and} \PY{n}{value} \PY{o}{\PYZlt{}}\PY{o}{=} \PY{l+m+mi}{10}\PY{p}{:}
                    \PY{n}{key\PYZus{}in} \PY{o}{=} \PY{k+kc}{False}
                \PY{k}{else}\PY{p}{:} 
                    \PY{n+nb}{print}\PY{p}{(}\PY{l+s+s2}{\PYZdq{}}\PY{l+s+se}{\PYZbs{}n}\PY{l+s+s2}{No, you are out of the input range!  Try again...}\PY{l+s+s2}{\PYZdq{}}\PY{p}{)}
            \PY{k}{except}\PY{p}{:}
                \PY{c+c1}{\PYZsh{} if the try: statement is false, i.e. not an integer, then the except:}
                \PY{c+c1}{\PYZsh{} block of code executes next.}
                \PY{n+nb}{print}\PY{p}{(}\PY{l+s+s2}{\PYZdq{}}\PY{l+s+se}{\PYZbs{}n}\PY{l+s+s2}{No, you typed in a string or decimal value!  Try again...}\PY{l+s+se}{\PYZbs{}n}\PY{l+s+s2}{\PYZdq{}}\PY{p}{)}
                  
        \PY{c+c1}{\PYZsh{} print the value}
        \PY{n+nb}{print}\PY{p}{(}\PY{l+s+s2}{\PYZdq{}}\PY{l+s+se}{\PYZbs{}n}\PY{l+s+s2}{You typed in }\PY{l+s+s2}{\PYZdq{}} \PY{o}{+} \PY{n+nb}{str}\PY{p}{(}\PY{n}{value}\PY{p}{)}\PY{p}{)}
\end{Verbatim}


    Python normally assumes that a variable entered via input() is a string.
We can convert an intended number to a value by the method used above.
If we need a floating point value, we would replace the

\begin{Shaded}
\begin{Highlighting}[]
\NormalTok{value }\OperatorTok{=} \BuiltInTok{int}\NormalTok{(}\BuiltInTok{input}\NormalTok{(}\StringTok{"Please type in an integer between 1 and 10 inclusive "}\NormalTok{))}
\end{Highlighting}
\end{Shaded}

with

\begin{Shaded}
\begin{Highlighting}[]
\NormalTok{value }\OperatorTok{=} \BuiltInTok{float}\NormalTok{(}\BuiltInTok{input}\NormalTok{(}\StringTok{"Please type in an integer between 1 and 10 inclusive "}\NormalTok{))}
\end{Highlighting}
\end{Shaded}

We've used error trapping for keyboard input, but as previously stated,
error trapping is useful in any situation where errors, whether input or
programmed, might possibly occur.

\hypertarget{using-a-while-loop-to-slice-strings}{%
\subsubsection{Using a while loop to slice
strings}\label{using-a-while-loop-to-slice-strings}}

String variables can be considered as a list of characters. As such, we
can slice strings just as we did with lists.

    \begin{Verbatim}[commandchars=\\\{\}]
{\color{incolor}In [{\color{incolor} }]:} \PY{c+c1}{\PYZsh{} enter a string}
        \PY{n}{str\PYZus{}var} \PY{o}{=} \PY{n+nb}{input}\PY{p}{(}\PY{l+s+s2}{\PYZdq{}}\PY{l+s+s2}{Type in the name of your favorite city \PYZhy{} }\PY{l+s+s2}{\PYZdq{}}\PY{p}{)}
        
        \PY{c+c1}{\PYZsh{} print the name one character at a time, but first, get the string length}
        \PY{n}{length} \PY{o}{=} \PY{n+nb}{len}\PY{p}{(}\PY{n}{str\PYZus{}var}\PY{p}{)}
        
        \PY{c+c1}{\PYZsh{} start counting with the 0 or first character}
        \PY{n}{count} \PY{o}{=} \PY{l+m+mi}{0}
        
        \PY{c+c1}{\PYZsh{} print the individual letters in a column}
        \PY{k}{while} \PY{n}{count} \PY{o}{\PYZlt{}} \PY{n}{length}\PY{p}{:}
            \PY{n+nb}{print}\PY{p}{(}\PY{n}{str\PYZus{}var}\PY{p}{[}\PY{n}{count}\PY{p}{]}\PY{p}{)}
            \PY{n}{count} \PY{o}{+}\PY{o}{=} \PY{l+m+mi}{1}
                  
        \PY{c+c1}{\PYZsh{} space}
        \PY{n+nb}{print}\PY{p}{(}\PY{l+s+s1}{\PYZsq{}}\PY{l+s+s1}{\PYZsq{}}\PY{p}{)}
                  
        \PY{c+c1}{\PYZsh{} print the individual letters in a column in reverse}
        
        \PY{c+c1}{\PYZsh{} this looks a bit odd, but remember we start counting from 0}
        \PY{c+c1}{\PYZsh{} so the last character is numbered one less than the length}
        \PY{c+c1}{\PYZsh{} i.e. if the string is 6 characters long, the last position = 5}
        
        \PY{n}{count} \PY{o}{=} \PY{n}{length} \PY{o}{\PYZhy{}} \PY{l+m+mi}{1}
        
        \PY{c+c1}{\PYZsh{} we\PYZsq{}ve got to to all the way to 0}
        \PY{k}{while} \PY{n}{count} \PY{o}{\PYZgt{}}\PY{o}{=} \PY{l+m+mi}{0}\PY{p}{:}
            \PY{n+nb}{print}\PY{p}{(}\PY{n}{str\PYZus{}var}\PY{p}{[}\PY{n}{count}\PY{p}{]}\PY{p}{)}
            \PY{n}{count} \PY{o}{\PYZhy{}}\PY{o}{=} \PY{l+m+mi}{1}
                  
        \PY{c+c1}{\PYZsh{} print the last two characters in the string}
        \PY{n+nb}{print}\PY{p}{(}\PY{l+s+s2}{\PYZdq{}}\PY{l+s+se}{\PYZbs{}n}\PY{l+s+s2}{The last two characters are }\PY{l+s+s2}{\PYZdq{}} \PY{o}{+} \PY{n}{str\PYZus{}var}\PY{p}{[}\PY{o}{\PYZhy{}}\PY{l+m+mi}{2}\PY{p}{:}\PY{p}{]}\PY{p}{)}
        
        \PY{n+nb}{print}\PY{p}{(}\PY{l+s+s1}{\PYZsq{}}\PY{l+s+s1}{\PYZsq{}}\PY{p}{)}
        
        \PY{c+c1}{\PYZsh{} now print the name on a single line one character at a time}
        
        
        \PY{c+c1}{\PYZsh{} start counting with the 0 or first character}
        \PY{n}{count} \PY{o}{=} \PY{l+m+mi}{0}
        
        \PY{c+c1}{\PYZsh{} print the individual letters in a line}
        \PY{k}{while} \PY{n}{count} \PY{o}{\PYZlt{}} \PY{n}{length}\PY{p}{:}
            \PY{n+nb}{print}\PY{p}{(}\PY{n}{str\PYZus{}var}\PY{p}{[}\PY{n}{count}\PY{p}{]}\PY{p}{,} \PY{n}{end} \PY{o}{=} \PY{l+s+s1}{\PYZsq{}}\PY{l+s+s1}{\PYZsq{}}\PY{p}{)}
            \PY{n}{count} \PY{o}{+}\PY{o}{=} \PY{l+m+mi}{1}
            
        \PY{c+c1}{\PYZsh{} we\PYZsq{}ve stripped the newline from the end of the print statement with end = \PYZsq{}\PYZsq{}}
        \PY{c+c1}{\PYZsh{} if we don\PYZsq{}t add an empty print statement or add a newline we\PYZsq{}ll keep}
        \PY{c+c1}{\PYZsh{} printing on the same line.}
        
        \PY{n+nb}{print}\PY{p}{(}\PY{l+s+s1}{\PYZsq{}}\PY{l+s+s1}{\PYZsq{}}\PY{p}{)}
        
        \PY{c+c1}{\PYZsh{} and now reverse the letters in a line}
        \PY{n}{count} \PY{o}{=} \PY{n}{length} \PY{o}{\PYZhy{}} \PY{l+m+mi}{1}
        
        \PY{c+c1}{\PYZsh{} we\PYZsq{}ve got to to all the way to 0}
        \PY{k}{while} \PY{n}{count} \PY{o}{\PYZgt{}}\PY{o}{=} \PY{l+m+mi}{0}\PY{p}{:}
            \PY{n+nb}{print}\PY{p}{(}\PY{n}{str\PYZus{}var}\PY{p}{[}\PY{n}{count}\PY{p}{]}\PY{p}{,} \PY{n}{end} \PY{o}{=} \PY{l+s+s1}{\PYZsq{}}\PY{l+s+s1}{\PYZsq{}}\PY{p}{)}
            \PY{n}{count} \PY{o}{\PYZhy{}}\PY{o}{=} \PY{l+m+mi}{1}
\end{Verbatim}


    \hypertarget{try-it-yourself}{%
\subsubsection{Try It Yourself}\label{try-it-yourself}}

Now for a little challenge\ldots{}

\begin{itemize}
\tightlist
\item
  Write a Python program that accepts string input from the keyboard.
\item
  Print the input, whether a single name or sentence, in reverse order
  one letter at a time.
\item
  Have the program continue to accept input (looping) until the user
  types `quit'.
\item
  No error trapping needed. Accept any input. You will see that numbers
  work as well as letters.
\end{itemize}

    \hypertarget{more-functions}{%
\subsubsection{More functions}\label{more-functions}}

Let's write a function to determine whether or not an integer is prime.
This script looks like a longer program and it is, but I've used a lot
of remark statements to help me remember how the program was written and
what I was thinking at the time.

    \begin{Verbatim}[commandchars=\\\{\}]
{\color{incolor}In [{\color{incolor} }]:} \PY{c+c1}{\PYZsh{} isprime function}
        
        \PY{k}{def} \PY{n+nf}{isprime}\PY{p}{(}\PY{n}{n}\PY{p}{)}\PY{p}{:}
            
            \PY{c+c1}{\PYZsh{} prime acts as a flag to let us know if we have a prime or not}
            \PY{c+c1}{\PYZsh{} we start out with prime = True}
            
            \PY{n}{prime} \PY{o}{=} \PY{k+kc}{True}
            
            \PY{c+c1}{\PYZsh{} we need to divide by all integers up to a certain point }
            \PY{c+c1}{\PYZsh{} to see if any are divisors of our test value.  We know}
            \PY{c+c1}{\PYZsh{} all numbers \PYZgt{} 2 can\PYZsq{}t be prime, so we start with 3 and}
            \PY{c+c1}{\PYZsh{} then use odd numbers}
            
            \PY{n}{divisor} \PY{o}{=} \PY{l+m+mi}{3}
            
            \PY{c+c1}{\PYZsh{} we will check divisors up to one less than half the test value}
            \PY{c+c1}{\PYZsh{} actually, we only need to check up to the square root of the}
            \PY{c+c1}{\PYZsh{} test value... n**(1/2) Why?}
            
            \PY{k}{while} \PY{n}{divisor} \PY{o}{\PYZlt{}} \PY{p}{(}\PY{n}{n}\PY{o}{/}\PY{l+m+mi}{2}\PY{p}{)}\PY{p}{:}
                
                \PY{c+c1}{\PYZsh{} the \PYZpc{} is the modular or remainder division symbol}
                \PY{c+c1}{\PYZsh{} it returns only the remainder in a division problem}
                \PY{c+c1}{\PYZsh{} if the remainder == 0, then we have a divisor and the}
                \PY{c+c1}{\PYZsh{} test value is NOT prime}
                
                \PY{k}{if} \PY{n}{n}\PY{o}{\PYZpc{}}\PY{k}{divisor} == 0:
                    
                    \PY{c+c1}{\PYZsh{} if we have a divisor, flag it by changing noprime to True}
                    \PY{n}{prime} \PY{o}{=} \PY{k+kc}{False}
                    
                    \PY{c+c1}{\PYZsh{} get out of the while loop... we are done}
                    \PY{k}{break}
                    
                \PY{k}{else}\PY{p}{:}
                    \PY{c+c1}{\PYZsh{} add two to the divisor and try again.  Since we started with}
                    \PY{c+c1}{\PYZsh{} 3, this statement runs through the odd numbers}
                    \PY{n}{divisor} \PY{o}{+}\PY{o}{=} \PY{l+m+mi}{2}
        
            \PY{c+c1}{\PYZsh{} now the moment of truth. If we went through all of the potential divisors}
            \PY{c+c1}{\PYZsh{} and did not change the value of prime from True to False, we have a prime number}
            
            \PY{k}{if} \PY{n}{prime} \PY{o}{==} \PY{k+kc}{True}\PY{p}{:}
                \PY{n+nb}{print}\PY{p}{(}\PY{n+nb}{str}\PY{p}{(}\PY{n}{n}\PY{p}{)} \PY{o}{+} \PY{l+s+s2}{\PYZdq{}}\PY{l+s+s2}{ is prime.}\PY{l+s+s2}{\PYZdq{}}\PY{p}{)}
            
            \PY{c+c1}{\PYZsh{} and if we DID find a divisor, prime was changed to False}
            \PY{c+c1}{\PYZsh{} and we\PYZsq{}ll print accordingly}
            
            \PY{k}{else}\PY{p}{:}
                \PY{n+nb}{print}\PY{p}{(}\PY{n+nb}{str}\PY{p}{(}\PY{n}{n}\PY{p}{)} \PY{o}{+} \PY{l+s+s2}{\PYZdq{}}\PY{l+s+s2}{ is composite.}\PY{l+s+s2}{\PYZdq{}}\PY{p}{)}
            
        
        \PY{c+c1}{\PYZsh{} test value }
            
        \PY{n}{isprime}\PY{p}{(}\PY{l+m+mi}{45587}\PY{p}{)}
\end{Verbatim}


    \begin{Verbatim}[commandchars=\\\{\}]
{\color{incolor}In [{\color{incolor} }]:} \PY{c+c1}{\PYZsh{} try your own test value here}
\end{Verbatim}


    Note: We can use the \% (mod or remainder division) to determine whether
or not a number is even or odd. Here is an example code block for that
purpose:

\begin{Shaded}
\begin{Highlighting}[]
\CommentTok{# assume the variable n is some integer}
\ControlFlowTok{if}\NormalTok{ n}\OperatorTok{%}\DecValTok{2} \OperatorTok{==} \DecValTok{0}\NormalTok{:}
    \BuiltInTok{print}\NormalTok{(}\StringTok{'even'}\NormalTok{)}
\ControlFlowTok{else}\NormalTok{:}
    \BuiltInTok{print}\NormalTok{(}\StringTok{'odd'}\NormalTok{)}
\end{Highlighting}
\end{Shaded}

Now let's write a square root function. I know Python can do this on its
own, but perhaps we can put one together using what we've already
learned. The following algorithm (computer code that performs a series
of calculations) is based on Newton's method. It is several hundred
years old (Newton's method, not the code), but is still useful.

    \begin{Verbatim}[commandchars=\\\{\}]
{\color{incolor}In [{\color{incolor} }]:} \PY{c+c1}{\PYZsh{} define a square root function using Newton\PYZsq{}s method}
        \PY{c+c1}{\PYZsh{} you can look this up online for a better explanation}
        
        \PY{k}{def} \PY{n+nf}{sqr}\PY{p}{(}\PY{n}{n}\PY{p}{)}\PY{p}{:}
            
            \PY{c+c1}{\PYZsh{} start with a guess of 0.5*n or 1/2 of the number}
            \PY{n}{guess} \PY{o}{=} \PY{l+m+mf}{0.5} \PY{o}{*} \PY{n}{n}
            
            \PY{c+c1}{\PYZsh{} make a better guess by taking the average of}
            \PY{c+c1}{\PYZsh{} the guess and the number divided by the guess}
            \PY{n}{better} \PY{o}{=} \PY{l+m+mf}{0.5} \PY{o}{*} \PY{p}{(}\PY{n}{guess} \PY{o}{+} \PY{n}{n}\PY{o}{/}\PY{n}{guess}\PY{p}{)}
            
            \PY{c+c1}{\PYZsh{} as long as the guess and the better value are}
            \PY{c+c1}{\PYZsh{} different (not equal or !=), keep on working}
            \PY{k}{while} \PY{n}{better} \PY{o}{!=} \PY{n}{guess}\PY{p}{:}
                
                \PY{c+c1}{\PYZsh{} replace guess with the better value and try again}
                \PY{n}{guess} \PY{o}{=} \PY{n}{better}
                
                \PY{c+c1}{\PYZsh{} make a better guess by taking the average of}
                \PY{c+c1}{\PYZsh{} guess and the number divided by guess}
                \PY{n}{better} \PY{o}{=} \PY{l+m+mf}{0.5} \PY{o}{*} \PY{p}{(}\PY{n}{guess} \PY{o}{+} \PY{n}{n}\PY{o}{/}\PY{n}{guess}\PY{p}{)}
                
            \PY{k}{return} \PY{n}{guess}
        
        \PY{c+c1}{\PYZsh{} test}
        \PY{n}{sqr}\PY{p}{(}\PY{l+m+mi}{2}\PY{p}{)}
            
\end{Verbatim}


    \begin{Verbatim}[commandchars=\\\{\}]
{\color{incolor}In [{\color{incolor} }]:} \PY{c+c1}{\PYZsh{} try your own test values here}
\end{Verbatim}


    \hypertarget{go-to-assignment-3}{%
\subsubsection{Go to Assignment 3}\label{go-to-assignment-3}}


    % Add a bibliography block to the postdoc
    
    
    
    \end{document}
